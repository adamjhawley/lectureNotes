% Created 2019-03-15 Fri 14:22
% Intended LaTeX compiler: pdflatex
\documentclass[11pt]{article}
\usepackage[utf8]{inputenc}
\usepackage[T1]{fontenc}
\usepackage{graphicx}
\usepackage{grffile}
\usepackage{longtable}
\usepackage{wrapfig}
\usepackage{rotating}
\usepackage[normalem]{ulem}
\usepackage{amsmath}
\usepackage{textcomp}
\usepackage{amssymb}
\usepackage{capt-of}
\usepackage{hyperref}
\usepackage{turnstile}
\author{Adam Hawley}
\date{\today}
\title{Lecture 1: Logical Agents}
\hypersetup{
 pdfauthor={Adam Hawley},
 pdftitle={Lecture 1: Logical Agents},
 pdfkeywords={},
 pdfsubject={},
 pdfcreator={Emacs 25.2.2 (Org mode 9.2.1)}, 
 pdflang={English}}
\begin{document}

\maketitle
\tableofcontents


\section{Intro}
\label{sec:org8ac8ab8}
The slides are very similar to the content of the book.
See chapter 7.

\section{Knowledge Bases}
\label{sec:org21b6bcd}
\begin{description}
\item[{Knowledge base}] Set of \textbf{sentences} in a \textbf{formal} language.
\end{description}
The declarative approach to building an agent (or other system) is to tell it what it needs to know.
Then it can ask itself what to do and the answers should follow from the KB.
Agents can be viewed:
\begin{itemize}
\item At the knowledge level: what they know, regardless of how they are implemented
\item At the implementation level: data structures in KB and algorithms that manipulate them
\end{itemize}

\section{Logic in General}
\label{sec:org687c359}
\begin{itemize}
\item \textbf{Logics} are formal languages for representing information such that conclusions can be drawn.
\item \textbf{Syntax} defines the sentences in the language.
\item \textbf{Semantics} define the \emph{meaning} of the sentences; i.e define truth of a sentence in a world.
\end{itemize}

\section{Entailment}
\label{sec:org34b83f8}
\begin{itemize}
\item \textbf{Entailment} means that one thing follows from another:
\end{itemize}
\begin{equation}
KB \models \alpha
\end{equation}
This means knowledge base \(KB\) entails sentence \(\alpha\) if and only if \(\alpha\) is true in all worlds where \(KB\) is true.
E.g. \(x + y = 4\) entails \(4 = x + y\)
Entailment is a relationship between sentences (i.e. syntax) that is based on semantics.

\section{Models}
\label{sec:org54e56c1}
Logicians typically think of \textbf{models}, which are formally structured worlds with respect to which truth can be evaluated.
We say \emph{m} \textbf{is a model of} a sentence \(\alpha\) if \(\alpha\) is true in \emph{m}.
M(\(\alpha\)) is the set of all models of \(\alpha\)

\section{Inference}
\label{sec:orgb6ce7cf}
In logic an inference is a procedure by which you can deduce that something does follow from something else.
\begin{itemize}
\item \(KB \vdash_i \alpha =\) sentence \(\alpha\) can be derived form \(KB\) by procedure \emph{i}.
\item \textbf{Soundness}: \emph{i} is sound if whenever \(KB \vdash_i \alpha\), it is also true that \(KB \models \alpha\).
\item \textbf{Completeness}: \emph{i} is complete if whenever \(KB \models \alpha\), it is also true that \(KB \vdash_i \alpha\).
\end{itemize}

\section{Propositional Logic: Syntax}
\label{sec:orgfc5d0b1}
Propositional logic is the simples logic.
The proposition sumbold \emph{P\textsubscript{1}}, \emph{P\textsubscript{2}} etc are senetences.
If \emph{S} is a sentence, ¬S is also a sentence (through negation).
This also applies for the rules of:
\begin{itemize}
\item Conjuction
\item Disjunction
\item Implication
\item Biconditional
\end{itemize}

\section{Logical Equivalence}
\label{sec:org06d2780}
Two sentences are \textbf{logically equivalent} iff true in the same models:
\begin{itemize}
\item \(\alpha \equiv \beta\) if and only if \(\alpha \models \beta\) and \(\beta \models \alpha\)
\end{itemize}

\section{Validity \& Satisfiability}
\label{sec:orge996c99}
A sentence is \textbf{valid} if it is true in all models (e.g \(True\), \(A\lor\neg A\), \(A\implies A\)).
Validity is connected to inference via the \textbf{Deduction Theorem}:
\begin{itemize}
\item \(KB \models \alpha\) if and only f \((KB \implies \alpha)\) is valid.
\end{itemize}
A sentence is \textbf{satisfiable} if it is true in \emph{some} model (e.g. \(A\lor B\)).
A sentence is \textbf{unsatisfiable} if it is true in \emph{no} models (e.g. \(A\land \neg A\))
Satisfiability is connected to inference via the following:
\begin{itemize}
\item \(KB \models \alpha\) if and only if \((KB \land \neg \alpha)\) is unsatisfiable, i.e. prove \(\alpha\) by \emph{reductio ad absurdum}.
\end{itemize}

\section{Proof Methods}
\label{sec:org12fc565}
\end{document}
