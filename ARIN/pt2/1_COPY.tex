% Created 2019-02-21 Thu 23:48
% Intended LaTeX compiler: pdflatex
\documentclass[11pt]{article}
\usepackage[utf8]{inputenc}
\usepackage[T1]{fontenc}
\usepackage{graphicx}
\usepackage{grffile}
\usepackage{longtable}
\usepackage{wrapfig}
\usepackage{rotating}
\usepackage[normalem]{ulem}
\usepackage{amsmath}
\usepackage{textcomp}
\usepackage{amssymb}
\usepackage{capt-of}
\usepackage{hyperref}
\author{Adam Hawley}
\date{\today}
\title{Lecture 1: Introduction to Supervised Learning}
\hypersetup{
 pdfauthor={Adam Hawley},
 pdftitle={Lecture 1: Introduction to Supervised Learning},
 pdfkeywords={},
 pdfsubject={},
 pdfcreator={Emacs 26.1 (Org mode 9.2)}, 
 pdflang={English}}
\begin{document}

\maketitle
\tableofcontents


\section{Organisation of this Section of ARIN}
\label{sec:org4ef6da0}
\begin{itemize}
\item 6 Lectures of this section
\item 2 exercise problem sheets (formative):
\begin{itemize}
\item Hand in solutions in groups of 3-4 to receive feedback
\item Solutions will be posted on the VLE
\end{itemize}
\item Two Q\&A sessions attendance voluntary.
\item Recommended text: Tom Mitchell, "Machine Learning"
\end{itemize}

\section{What is Machine Learning?}
\label{sec:orga9bb5aa}
   \emph{A computer program is said to learn from experience E with respect to some class of tasks T and performance measure P, if its performance at tasks in T, as measured by P, improves with experience E.}
[Mitchell 97]

\section{Types of Machine Learning}
\label{sec:orgfdd196b}
\subsection{Supervised Learning}
\label{sec:org94f2cc5}
This is where the learning agent receives training examples and corresponding labels provided by a supervisor.
The goal is summarised as "Given a new example, what is its label?".

\subsection{Reinforcement Learning}
\label{sec:org66fde40}
Training experience are state-action paris with the correspondingn numerical reward.
The goal is summarised here as "Learn a behaviour that maximises cumulative reward.

\subsection{Unsupervised Learning}
\label{sec:orgffb7ffe}
Unstructured set of examples.
Goal: discover patterns/structures in the data.
Focus: clustering.

\section{Supervised Learning}
\label{sec:org82fc0d5}
\subsection{Issues}
\label{sec:orgb728122}
\begin{itemize}
\item Where to get training examples from? (e.g medical data can be difficult to acquire because of data privacy)
\item How to represent examples?
\item How to represent classification procedure (hypothesis)?
\item Which learning method to use?
\item How to evaluate the result?
\end{itemize}

\subsection{Learning (Generalisation) Bias}
\label{sec:orgd389894}
\emph{Definition}: Preference relation between legal hypotheses.

Hypothesis with zero error on training data is not necessarily the best (noise!).
\textbf{Occam's razor}: the simpler hypothesis is the better one.
No supervised learning without some generalisation (can be caused by language or learning bias).
\textbf{Language bias} is bias that comes from your hypothesis representation.
\textbf{Learning bias} comes from the machine learning algorithm.
\end{document}
