% Created 2019-03-08 Fri 20:00
% Intended LaTeX compiler: pdflatex
\documentclass[11pt]{article}
\usepackage[utf8]{inputenc}
\usepackage[T1]{fontenc}
\usepackage{graphicx}
\usepackage{grffile}
\usepackage{longtable}
\usepackage{wrapfig}
\usepackage{rotating}
\usepackage[normalem]{ulem}
\usepackage{amsmath}
\usepackage{textcomp}
\usepackage{amssymb}
\usepackage{capt-of}
\usepackage{hyperref}
\author{Adam Hawley}
\date{\today}
\title{ML Section 1.4: Evaluation of Supervised Learning}
\hypersetup{
 pdfauthor={Adam Hawley},
 pdftitle={ML Section 1.4: Evaluation of Supervised Learning},
 pdfkeywords={},
 pdfsubject={},
 pdfcreator={Emacs 26.1 (Org mode 9.2)}, 
 pdflang={English}}
\begin{document}

\maketitle
\tableofcontents


\section{Hypothesis Evaluation}
\label{sec:org5c85fb4}
General questions:
\begin{itemize}
\item How can one estimate the performance of a learned hypothesis on future data?
\item How good is the estimate?
\item Comparative performance evaluations.
\end{itemize}
\textbf{Formally:} Given a hypothesis \emph{h} and a data sample containing \emph{n} examples drawn at random according to the distribution \emph{D}, what is the best estimate of the accuracy of \emph{h} over future instances drawn from the same distribution?
What is the probable error in this accuracy estimate?

\section{Evaluation Problems}
\label{sec:org6a1e430}
\begin{itemize}
\item Limited samples of data may be misleading (e.g. prime numbers and data set = \{3,5,7\} leads to hypothesis of odd numbers).
\item Observed accuracy on training data is often too optimistic (e.g. due to overfitting).
\item Solution: use independent test examples.
\item Problem: estimate may still depend on the specific makeup of the set of training/test examples.
\end{itemize}

\section{Preliminary Definitions:}
\label{sec:org1de180a}
\begin{description}
\item[{\emph{f}}] The target customisation function to be learned (f:Examples \(\rightarrow\) Categories).
\item[{\emph{h}}] The hypothesis learned (h: Examples \(\rightarrow\) Categories).
\item[{\emph{S}}] Data sample of size \emph{n}.
\item[{\emph{D}}] Probability distribution over all data points.
\item[{Sample Error}] 
\end{description}
\begin{equation}
error_s(h) = \frac{1}{2}\sum\limits_{x\in S} \delta(f(x),h(x)
\end{equation}
Where:
\begin{equation}
\delta(y,z) = \text{1 if }y\neq z\text{, and 0 otherwise.}
\end{equation}
\begin{description}
\item[{True Error}] 
\end{description}
\begin{equation}
error)D(h) = Pr_{x \in D}[f(x)\neq h(x)]
\end{equation}

\section{Confidence Intervals}
\label{sec:org79a8da5}
\end{document}
