\documentclass{article}

\usepackage{amsmath}

\title{Lecture 3: Uninformed Search}
\author{Adam Hawley}

\begin{document}

\maketitle
\tableofcontents
\newpage

\section{Implementation of Search}

\subsection{State vs. Node}
\begin{itemize}
	\item A \textbf{state} is a representation of a physical configuration
	\item A \textbf{node} is a data structure which constitutes part of a search tree including state, parent node, action, path cost and depth.
\end{itemize}

\subsection{Uninformed vs. Informed}
\begin{itemize}
	\item Uninformed (Blind) Search Algorithms: In making this decision, these look only at the structure of the search tree and not at the states inside the nodes.
	\item Informed (Heuristic) Search Algorithms: In making this decision these look at the states inside the nodes.
\end{itemize}

\section{Evaluating Search Strategies}
Strategies are evaluated along the following dimensions:
\begin{itemize}
	\item Completeness: does it always find a solution if one exists
	\item Time complexity: number of nodes generated (not expanded)
	\item Space complexity: maximum number of nodes in memory
	\item Optimality: does it always find a least-cost solution?
\end{itemize}
Time and space complexity will use the following parameters:
\begin{itemize}
	\item \textbf{b}: maximum branching factor of the search tree
	\item \textbf{d}: depth of the least-cost solution (root is of depth 0)
	\item \textbf{m}: maximum depth of the search tree (may be $\infty$)
\end{itemize}

\section{Breadth-First Search}
Expand the shallowest unexpanded node.

Implementation: \textit{fringe} is a FIFO queue, so new nodes go at the end and nodes are selected from the front.

Properties:
\begin{itemize}
	\item Complete: Yes (if b is finite)
	\item Time Complexity: $1+b+b^2+b^3+...+b^d = O(b^{d+1}) = O(b.b^d) = \text{(If b is constant)} O({b^d})$
	\item Space Complexity: $O(b^{d+1})$ or $O(b^d)$ if only fringe is in memory
	\item Optimal: Yes, if all operators have the same cost
\end{itemize}
Space is a bigger problem than time.
What if we want to find the optimal solution and operators have different costs?

\section{Uniform-Cost Search}
Expand the least-cost unexpended node.
Implementation: finge is queue ordered by increasing path cost.
Nodes are selected from the front.

Properties:
\begin{itemize}
	\item Complete: Yes, if step cost $\le \epsilon$
	\item Time Complexity: \# of nodes with $g \le C^*$, where $C^*$ is the cost of the optimal solution, so $O(b^{\lceil\frac{C^*}{\epsilon}\rceil})$. Or $O(b^d)$.
	\item Space Complexity: \# of nodes with g $\le$ cost of optimal solution, $O(b^{\lceil\frac{C^*}{\epsilon}\rceil})$. Or $O(b^d)$.
	\item Optimal: Yes, since nodes are expanded in increasing order of $g(n)$.
\end{itemize}

\section{Depth-First Search}
Expand the deepest unexpanded node.
Implementation: Fringe is a LIFO queue (or stack), so new nodes are put at front and nodes removed from the front.

Properties:
\begin{itemize}
	\item Complete: No, could run forever if a tree has infinite depth. Complete in finite spaces though.
	\item Time Complexity: $O(b^m)$ (Size of the tree)
	\item Space Complexity: $O(bm)$ (Linear space)
	\item Optimal: No.
\end{itemize}

\section{Depth-Limited Search}
Depth-first search with depth limit $l$.
That is, nodes whose depth is greater than $l$ are ignored.

Implementation: same as depth-first search but if a node has depth $l$ then it is not expanded.

Properties:
\begin{itemize}
	\item Complete: No, solution not found if $d>l$
	\item Time complexity: $O(b^l)$
	\item Space complexity: $O(bl)$ 
	\item Optimal: No.
\end{itemize}

\section{Iterative Deepening Search}
To overcome the incompleteness of depth-limited search, if no solution is found, redo the search with an increased depth limit.

Properties:
\begin{itemize}
	\item Complete: Yes
	\item Time complexity: $(d+1)b^0 + db^1 + (d-1)b^2 + ... + b^d = O(b^d)$
	\item Space complexity: $O(bd)$
	\item Optimal: Yes, if step cost = 1
\end{itemize}

See slide 54 for a table containing all of the properties.

\section{Summary}
\begin{itemize}
	\item Problem represenation is important.
		Some representations can have much smaller search trees.
	\item Variety of uninformed search strategies.
	\item Iterative deepening search uses only linear space and not much more time than other uninformed algorithms.
	\item It is often important to eliminate repeated states.
\end{itemize}
\end{document}
