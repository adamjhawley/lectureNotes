% Created 2019-04-18 Thu 19:24
% Intended LaTeX compiler: pdflatex
\documentclass[11pt]{article}
\usepackage[utf8]{inputenc}
\usepackage[T1]{fontenc}
\usepackage{graphicx}
\usepackage{grffile}
\usepackage{longtable}
\usepackage{wrapfig}
\usepackage{rotating}
\usepackage[normalem]{ulem}
\usepackage{amsmath}
\usepackage{textcomp}
\usepackage{amssymb}
\usepackage{capt-of}
\usepackage{hyperref}
\author{Adam Hawley}
\date{\today}
\title{Lecture 5: Inference in First-Order Logic (Resolution)}
\hypersetup{
 pdfauthor={Adam Hawley},
 pdftitle={Lecture 5: Inference in First-Order Logic (Resolution)},
 pdfkeywords={},
 pdfsubject={},
 pdfcreator={Emacs 26.1 (Org mode 9.2)}, 
 pdflang={English}}
\begin{document}

\maketitle
\tableofcontents


\section{Resolution in Propositional Logic}
\label{sec:org5f87c1a}
The resolution inference rule for propositional logic derives a new clause from two clauses.
See the example of resolution below:
\begin{equation}
\frac{p_{1,1} \lor p_{3,1}, ¬p_{1,1} \lor ¬p_{2,2}}{p_{3,1} \lor ¬p_{2,2}}
\end{equation}
\end{document}
