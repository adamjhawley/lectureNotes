% Created 2019-03-26 Tue 15:38
% Intended LaTeX compiler: pdflatex
\documentclass[11pt]{article}
\usepackage[utf8]{inputenc}
\usepackage[T1]{fontenc}
\usepackage{graphicx}
\usepackage{grffile}
\usepackage{longtable}
\usepackage{wrapfig}
\usepackage{rotating}
\usepackage[normalem]{ulem}
\usepackage{amsmath}
\usepackage{textcomp}
\usepackage{amssymb}
\usepackage{capt-of}
\usepackage{hyperref}
\author{Adam Hawley}
\date{\today}
\title{Lecture 3: SAT Solvers \& Local Search}
\hypersetup{
 pdfauthor={Adam Hawley},
 pdftitle={Lecture 3: SAT Solvers \& Local Search},
 pdfkeywords={},
 pdfsubject={},
 pdfcreator={Emacs 26.1 (Org mode 9.2)}, 
 pdflang={English}}
\begin{document}

\maketitle
\tableofcontents


\section{Giving Up On Completeness}
\label{sec:orgfdda17d}
SAT is NP-Complete and is hard.
One option for the impatient is to maintain soundness but give up on completeness.
So search for a satisfying assignment if we find one then CNF is satisfiable and if not then we don't know either way.

\section{Local Search}
\label{sec:org3c757ab}
Look for \emph{good} states \emph{near} the current state.
Goal states are particularly good!
Don't bother remembering where you have been so little memory is required.

\section{Local Search for SAT}
\label{sec:org2c79fdd}
\begin{itemize}
\item States are completely defined models, not partially-defined ones.
\item Search is for a satisfying model: these are goal states.
\item Natrual evaluation function is: number of unsatisfied clauses.
\item Clearly the goal is to minimise this function.
\end{itemize}

\subsection{Min-Conflicts Step}
\label{sec:org6abcf7a}
Choose a symbol such that \textbf{flipping} its truth-value minimises the number of unsatisfied clauses.
Only one symbol is affected and this is the reason that this method is considered \emph{local}.

\subsection{Random Walk Step}
\label{sec:org2ad3ad1}
In a random walk step, a symbol is chosen at random.
Its truth-value is flipped.
This can help the searchh escape a local minimum.

\subsection{WalkSAT Algorithm}
\label{sec:org5701be4}
For each local move only flip symbols appearing in an unsatisfied clause.
Choose random walk step with probability p and min-conflict step with probability 1-p.
p is usually set to be about 0.5.

\subsubsection{Features of the WalkSAT Algorithm}
\label{sec:orgddfb1c1}
\begin{itemize}
\item If p>0, max\textunderscore flips = \(\infty\) and the CNF is satisfiable then WalkSAT will (eventually) find a model.
\item But if max\textunderscore flips  = \(\infty\) and the CNF is unsatisfiable it will never terminate.
\end{itemize}
WalkSAT is therefore a good option when we know (or at least suspect) that the CNF is satisfiable, but want to find a satisfying model.

\section{Under and Over-Constrained SAT Problems}
\label{sec:org6be04df}
More symbols means more models, increading the \emph{chance} of find = \(\infty\) and the CNF is unsatisfiable it will never terminate.
WalkSAT is therefore a good option when we know (or at least suspect) that the CNF is satisfiable, but want to find a satisfying model.

\section{Under and Over-Constrained SAT Problems}
\label{sec:orgef2ce55}
More symbols means more models, increasing the \emph{chance} of finding one that satisfies the CNF.
More clauses means fewer models, decreasing the \emph{chance} of finding one that satisfies the CNF.
This is why the \(\frac{clause}{symbol}\) ratio matters.
If it is low then the problem is underconstrained, if high then it is overconstrained.

See lecture slides for proof.
\end{document}
