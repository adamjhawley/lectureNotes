% Created 2019-04-14 Sun 13:50
% Intended LaTeX compiler: pdflatex
\documentclass[11pt]{article}
\usepackage[utf8]{inputenc}
\usepackage[T1]{fontenc}
\usepackage{graphicx}
\usepackage{grffile}
\usepackage{longtable}
\usepackage{wrapfig}
\usepackage{rotating}
\usepackage[normalem]{ulem}
\usepackage{amsmath}
\usepackage{textcomp}
\usepackage{amssymb}
\usepackage{capt-of}
\usepackage{hyperref}
\author{Adam Hawley}
\date{\today}
\title{Lecture 4: First-Order Logic}
\hypersetup{
 pdfauthor={Adam Hawley},
 pdftitle={Lecture 4: First-Order Logic},
 pdfkeywords={},
 pdfsubject={},
 pdfcreator={Emacs 26.1 (Org mode 9.2)}, 
 pdflang={English}}
\begin{document}

\maketitle
\tableofcontents


\section{Pros \& Cons of Propositional Logic}
\label{sec:org86d6434}
\subsection{Pros}
\label{sec:org748b300}
\begin{itemize}
\item Propositional logic is \textbf{declarative}: pieces of syntax correspond to facts.
\item Propositional logic allows partial/disjunctive/negated information (unlike most data structures and databases).
\item Propositional logic is \textbf{compositional} (i.e. meaning of \(B_{1.1}\land P_{1.2}\) is derived from meaning of \(B_{1.1}\) and of \(P_{1.2}\)).
\item Meaning in propositional logic is \textbf{context-independent}.
\end{itemize}

\subsection{Cons}
\label{sec:orgea65499}
Propositional logic has very limited expressive power (e.g. cannot say "pits cause breezes in adjacent squares" except by writing one sentence for each square).

\section{First-Order Logic}
\label{sec:orge97c7be}
Whereas propositional logic assumes world contains facts, first-order logic assumes the world contains:
\begin{description}
\item[{Objects}] People, houses, numbers, theories etc.
\item[{Relations}] Red, round, bogus, prime, etc.
\item[{Functions}] Father of, best friend, one more than etc.
\end{description}

\section{Atomic Sentences}
\label{sec:org408033b}
The simplest form of first-order logic is in \textbf{atomic sentences}.
\begin{description}
\item[{Atomic Sentence}] \(predicate(term_1, ..., term_n)\) or \(term_1=term_2\).
\item[{Term}] \$function(term\textsubscript{1}, \ldots{}, term\textsubscript{n}) or \(constant\) or \(variable\).
\end{description}

\section{Complex Sentences}
\label{sec:org3588f15}
Complex sentences are made from atomic sentences using connectives.

\section{Truth in First-Order Logic}
\label{sec:org4974a21}
Sentences are true with respect to a model and an interpretation.
Models contain \(\ge\) 1 objects (\textbf{domain elements}) and relations among them.

Interpretation specifies referents for:
\begin{itemize}
\item constant symbols \(\implies\) objects
\item predicate symbols \(\implies\) relations
\item function symbols \(\implies\) functional relations
\end{itemize}

\section{Problems with FOL}
\label{sec:org577f47b}
\subsection{Frame Problem}
\label{sec:org0414a7a}
Finding an elegant way to handle non-change rather than repeated frame axioms.

\subsection{Qualification Problem}
\label{sec:orgf0b48af}
True descriptions of real actionss require endless caveats--- what if gold is slippery or nailed down or\ldots{}

\subsection{Ramification Problem}
\label{sec:org76e5a39}
Real actions have many secondary consequences--- what about dust on the gold, war and tear on gloves, etc.
\end{document}
