% Created 2019-02-05 Tue 15:37
% Intended LaTeX compiler: pdflatex
\documentclass[11pt]{article}
\usepackage[utf8]{inputenc}
\usepackage[T1]{fontenc}
\usepackage{graphicx}
\usepackage{grffile}
\usepackage{longtable}
\usepackage{wrapfig}
\usepackage{rotating}
\usepackage[normalem]{ulem}
\usepackage{amsmath}
\usepackage{textcomp}
\usepackage{amssymb}
\usepackage{capt-of}
\usepackage{hyperref}
\author{Adam Hawley}
\date{\today}
\title{Lecture 6: Coordination Pt.2}
\hypersetup{
 pdfauthor={Adam Hawley},
 pdftitle={Lecture 6: Coordination Pt.2},
 pdfkeywords={},
 pdfsubject={},
 pdfcreator={Emacs 26.1 (Org mode 9.2)}, 
 pdflang={English}}
\begin{document}

\maketitle
\tableofcontents


\section{Shared Variable Coordination}
\label{sec:org76b0c1a}
\subsection{Monitors}
\label{sec:orgc6b39f7}
A monitor is a control structure that provides \textbf{mutual exclusion}.
It can also be considered to be "An extension of the conditional critical sections".

\subsection{Pascal-FC Implementation of Monitors}
\label{sec:orgc133918}
See lecture for live coding.

Pascal-FC uses a low-level condition variable mechanism with three operators:
\begin{description}
\item[{delay}] immediately blocks caller task, and releases the monitor
\item[{resume}] makes at most one blocked task runnable again
\item[{empty}] return true if no blocked tasks exist
\end{description}

\subsection{Ada Implementation of Monitors}
\label{sec:orgae60d96}
In Ada, monitors are called \textbf{protected types}.
There are no condition variables like there are in Pascal-FC.
Instead, synchronisation comes from the use of barriers (guards).

Specifications of protected types can contain:
\begin{itemize}
\item Procedures executed under \textbf{mutual exclusion}
\item Pure functions that are \textbf{read only} can execute concurrently with other functions.
\item Entries specify \textbf{barriers} (or guards).
\item Private data of any type.
\end{itemize}

\subsubsection{Barriers}
\label{sec:org2303eb6}
Barriers are boolean expressions that must evaluate to \texttt{true} for the entry to execute.
If \texttt{false}, the task is blocked.
Barriers are re-evaluated on task entry and exit from the \textbf{protected object} (\textbf{PO}).
At most one task can proceed through the barrier.

\subsubsection{Attributes}
\label{sec:orgea96a9c}
Attributes in Ada give information about the behaviour of the program, e.g:
\begin{verbatim}
entry await when await `count = 4 is ...
\end{verbatim}
This releases one task (non-deterministically) when there are 4 waiting, otherwise blocks.
\end{document}
