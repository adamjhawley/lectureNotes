% Created 2019-02-27 Wed 21:07
% Intended LaTeX compiler: pdflatex
\documentclass[11pt]{article}
\usepackage[utf8]{inputenc}
\usepackage[T1]{fontenc}
\usepackage{graphicx}
\usepackage{grffile}
\usepackage{longtable}
\usepackage{wrapfig}
\usepackage{rotating}
\usepackage[normalem]{ulem}
\usepackage{amsmath}
\usepackage{textcomp}
\usepackage{amssymb}
\usepackage{capt-of}
\usepackage{hyperref}
\author{Adam Hawley}
\date{\today}
\title{Lecture 10: Message Passing 2}
\hypersetup{
 pdfauthor={Adam Hawley},
 pdftitle={Lecture 10: Message Passing 2},
 pdfkeywords={},
 pdfsubject={},
 pdfcreator={Emacs 26.1 (Org mode 9.2)}, 
 pdflang={English}}
\begin{document}

\maketitle
\tableofcontents


\section{Pascal-FC}
\label{sec:org4d6c208}
\begin{itemize}
\item Synchronous communication
\item Unlimited message types
\item Indirect naming via channels
\item Guarded select statements
\item Has an extended rendezvous mechanism
\end{itemize}

\section{Ada}
\label{sec:orgc95c3c3}
\begin{itemize}
\item \textbf{Remote invocation} communication model
\begin{itemize}
\item Can be used to provide \textbf{sychronous communication}
\end{itemize}
\item Unlimited message types
\item \textbf{Direct symmetric} naming via task names, and an entry defined for that task
\item Guarded select statements
\item Has extended rendezvous
\end{itemize}

\subsection{Ada Communication Model}
\label{sec:org9cc16bd}
Based on a \textbf{client/server} coordination model.
A \textbf{service} is a \textbf{public} \texttt{entry} in the server's task specification.
An \texttt{entry} declaration specifies the name, parameters and result types for the service.

\subsection{Other Facilities}
\label{sec:org2a9da9f}
\begin{itemize}
\item \texttt{`count} gives the number of tasks queued on an entry.
\item Entry families declare groups of entries
\item Nested accept statements allow multiple task coordination
\item A task executing in an \texttt{accept} can also execute an \texttt{entry} call
\end{itemize}

\subsection{\texttt{select}}
\label{sec:orge9b63ab}
The select statement comes in four forms:
\begin{verbatim}
select_statement ::= selective_accept
                     conditional_entry_call
                     timed_entry_call
                     asynchronous_select
\end{verbatim}
\subsubsection{\texttt{selective\_accept}}
\label{sec:org00b1dcc}
This allows the server to:
\begin{itemize}
\item wait (for more than one more rendezvous at a time)
\item time-out (if no rendezvous occurs within a specified time)
\item terminate (if no client can ever call an entry)
\end{itemize}

\section{Synchronisation}
\label{sec:org980b5bc}
\begin{itemize}
\item Both tasks must \emph{agree} to communicate
\item \textbf{Ready} task \textbf{waits} for the other task (blocked)
\item When both tasks are ready, client's arguments are passed to the server.
\item Server executed code inside the \texttt{accept} statement
\item Results returned to client at completion of \texttt{accept}
\item Tasks continue concurrently
\end{itemize}

I can't tell if there was actually more content or not\ldots{}
\end{document}
