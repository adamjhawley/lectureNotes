\documentclass{article}

\title{Lecture 5: Coordination 1}
\author{Adam Hawley}

\begin{document}

\maketitle

\section{Interacting Processes}
Tasks need to share data.
They do this by coordinating, here are some examples of tasks which require coordination:
\begin{itemize}
	\item One at a time through a critical section
	\item A starts X after B finishes Y
	\item Replicated tasks need to combine/compare results
	\item Work needs to be allocated to a manager
\end{itemize}
There are two approaches to interacting processes: \textbf{shared variables} and \textbf{message passing}.

\subsection{Mutual Exlusion --- Mutex}
\textbf{Critical sections} are code that must not be executed by more than one task at a time.
Unfortunately implementations of mutual exclusion are often complex and error prone.
They also do not easily generalise to $n$ tasks nor do they easilly generalise to more complex problems.

\subsection{Levels of Support}
\textbf{Simple primitive}:
\begin{itemize}
	\item Semaphores --- simple processes for guaranteeing mutual exclusion.
	\item Mutexes (normally provided by the runtime environment/OS so not discussed in detail).
\end{itemize}
\textbf{Control structures}:
\begin{itemize}
	\item Monitors which are normally provided by the language.
\end{itemize}

\subsubsection{Semaphores}
A semaphore is a non-negative integer together with two primitives: {\tt wait} and {\tt signal}.
On creation, a semaphore is initialised to 1 (in the simplest case).
\begin{itemize}
	\item {\tt signal} --- Atomically incrememnts a semaphore.
	\item {\tt wait} --- If the semaphore has a value greater than zero, decrements the value by 1.
If the semaphore is equal to 0 then the executing task \textbf{blocks}
\end{itemize}
Blocking is when a task is not runnable.
Tasks are unblocked when its semaphore becomes $>0$.
If multiple tasks are blocked on a semaphore, {\tt signal(sem)} will unblock one task chosen non-deterministically.
\end{document}
