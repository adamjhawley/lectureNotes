% Created 2019-02-22 Fri 16:13
% Intended LaTeX compiler: pdflatex
\documentclass[11pt]{article}
\usepackage[utf8]{inputenc}
\usepackage[T1]{fontenc}
\usepackage{graphicx}
\usepackage{grffile}
\usepackage{longtable}
\usepackage{wrapfig}
\usepackage{rotating}
\usepackage[normalem]{ulem}
\usepackage{amsmath}
\usepackage{textcomp}
\usepackage{amssymb}
\usepackage{capt-of}
\usepackage{hyperref}
\author{Adam Hawley}
\date{\today}
\title{Lecture 9: Message Passing 1}
\hypersetup{
 pdfauthor={Adam Hawley},
 pdftitle={Lecture 9: Message Passing 1},
 pdfkeywords={},
 pdfsubject={},
 pdfcreator={Emacs 26.1 (Org mode 9.2)}, 
 pdflang={English}}
\begin{document}

\maketitle
\tableofcontents


\section{Message-Based Coordination}
\label{sec:orge60e68a}
Shared-variable coordination is based on \emph{normal} variable access by multiple processes.
Message-passing coordination requires new primitives.
Abstractly, these are seen as:
\begin{itemize}
\item \texttt{send(message)}
\item \texttt{receive(message)}
\end{itemize}
A message can never be received (read) before it has been sent (written).
This means that \texttt{receive(message)} is potentially \textbf{blocking}.

\section{Evaluating Message-Based Coordination}
\label{sec:orgbb77c1f}
There are a number of ways to describe different approaches:
\begin{itemize}
\item Models for send and receive
\item How tasks are identified
\item The relationship between tasks
\item Types of messaages that can be sent
\end{itemize}

\section{Possible Models for Sending}
\label{sec:org6ff690c}
\subsection{Aynchronous}
\label{sec:org2340f1c}
Sender does not wait, \texttt{send()} returns \emph{immediately} but sent messages must be buffered.
\subsection{Synchronous}
\label{sec:org6c4db6f}
Sender blocks until receiver can receive, vice-versa.
Message can pass directly from sender to receiver, this means no buffer is necessary.
Sometimes this approach is called \emph{rendezvous}.
\subsection{Remote Invocation}
\label{sec:org71f589c}
Sender blocks until receiver is ready (Synchronous).
This time, when the message is received, the receiver will flag an acknowledgement or reply from the reviver.
This is also known as an \emph{extended rendezvous}.

\section{Task Naming}
\label{sec:org6515bf2}
How do senders and receivers refer to each other?
\begin{itemize}
\item Direct
\item Indirect
\item Symmetric
\item Asymmetric
\end{itemize}
\subsection{Direct}
\label{sec:orgdb9523f}
Tasks have names.
This is used to identify end points, e.g send(message) to Joe.
\subsection{Indirect}
\label{sec:org8516f61}
An intermediate is used (e.g. a \emph{mailbox} or a \emph{channel}).
These names can be statically named.
\subsection{Symmetric}
\label{sec:org48df0a6}
Sender specifies receiver and vice versa.
For example:
\begin{itemize}
\item \texttt{send(message) to task}
\item \texttt{receive(message) from task}
\end{itemize}
\subsection{Asymmetric}
\label{sec:orgbd3a5d0}
Send to named receiver, receive from any sender:
\begin{itemize}
\item \texttt{send(message) to task}
\item \texttt{receive(message)}
\end{itemize}

\section{Task Relationships}
\label{sec:org38d6cf7}
What is the relationship between senders and receivers?
\begin{itemize}
\item One-to-one (can be synchronous or buffered)
\item One-to-many
\begin{itemize}
\item A \textbf{multi-cast} communication
\item Sometimes inaccurately called a \emph{broadcast} which is one-to-everybody
\item Requires indirect naming
\end{itemize}
\item Many-to-many (multiple multi-casts)
\end{itemize}

\section{Message Types}
\label{sec:org3c20629}
These can be limited to a set of predefined types or unlimited (all types/classes can be communicated).
Low-level languages are more likely to limit message type.

\textbf{SEE LECTURE FOR OCCAM}
\end{document}
