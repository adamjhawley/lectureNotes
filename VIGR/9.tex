\documentclass{article}

\usepackage{amsmath}
\usepackage{gensymb}

\title{Lecture 9: Colour Image Representation}
\author{Adam Hawley}

\begin{document}

\maketitle
\section{Colour Image Pipeline}
\begin{itemize}
	\item Bayer colour filter arrays --- mosaics of colour
	\item Coordinated colour temperature transform (CCT) --- colours are adjusted based on light source illiminating the scene --- based on tables of chromacity for standard illumination
	\item Colour gamut -- depends on a specific device
	\item White balance adjustment
	\item Demosaicing --- interpolates the R,G,B components producing the full colour image
\end{itemize}

\subsection{Bayer Mosaics}
Bayer pattern sensors are composed of colour filtered sensors in a specific layout, using two green sensors for each red and blue sensor.

\subsection{Demosaicing}
Demosaicing is the process of turning the sensor output with 3 separate colour channels into one pixel per three colour values.
I.e going from 2 greens, 1 red and 1 blue to one pixel which represents them all.

\subsection{Colour Matching}
Colour matching functions work by combining red, green and blue to define the colour accuracy.
A particular colour (spectral wavelength) can be matched to filtered measurements of sensor values.
The matching is performed as relative values of colour primaries.
A different sensor could provide different colour values.

\subsection{Colour Gamut}
The colour gamut represents the colours which are reproduced by a specific media (the rest of the colours are clipped).
The gamut restrics which colours can be visualised.
Graphics artists should consider the display media gamut for the colours of his graphics.

\subsection{CIE}
The CIE chromacity diagram is obtained based on their perception.
Hue is defined on the  border.
Colour saturation decreases as you go towards the centre (location of white).

\section{Colour Systems}
\subsection{RGB}
The 3 types of cone in the human eyes are sensitive to 3 wavelengths: 630nm (red), 530nm (green) and 450nm (blue).
\begin{align*}
	C_\lambda = R\textbf{R} + G\textbf{G} + B\textbf{B}
\end{align*}
The values for \textbf{R,G,B} are defined by:
\begin{itemize}
	\item Standard NTSC phosphor
	\item Monitor/phosphur manufacturers
\end{itemize}
For two monitors 1 and 2 with different gamuts we can compute:
\begin{align*}
	\begin{bmatrix}
		Y \\ Y \\ Z
	\end{bmatrix}
	=
	\begin{bmatrix}
		X_r & X_g & X_b \\
		Y_r & Y_g & Y_b \\
		Y_r & Y_g & Y_b \\
	\end{bmatrix}
	\begin{bmatrix}
		R \\ G \\ B
	\end{bmatrix}
\end{align*}

\subsection{YIQ and CMY}
\subsubsection{YIQ}
Composite signal for TV defined by NTSC.
This colour system is based on CIE.
The Y parameter carries luminance (similar to the CIE).
I and Q contribute to hue and purity.
\begin{align*}
	\begin{bmatrix}
		Y \\ I \\ Q
	\end{bmatrix}
	= 
	\begin{bmatrix}
	0.299 & 0.587 & 0.144 \\
	0.596 & -0.275 & -0.321 \\
	0.212 & -0.528 & 0.311
	\end{bmatrix}
	\cdot
	\begin{bmatrix}
		R \\ G \\ B
	\end{bmatrix}
\end{align*}

\subsubsection{CMY}
Subtractive colour model used in printing.
Cyan ink subtracts red component of light, magenta subtracts green component and yellow subtracts the blue component of light.
The printing process uses 3 or 4 dots. (CMYK)
\begin{align*}
	\begin{bmatrix}
		C \\ M \\ Y
	\end{bmatrix}
	= 
	\begin{bmatrix}
		1 \\ 1 \\ 1
	\end{bmatrix}
	-
	\begin{bmatrix}
		R \\ G \\ B
	\end{bmatrix}
\end{align*}

\subsection{3D Polar Coordinates}
These spaces use a cylindrical (3D polar) coordinate system to encode the following three psycho-visual coordinates:
\begin{itemize}
	\item Hue (Dominant Colour Seen)
		\begin{itemize}
			\item Wavelength of the pure colour observed in the signal.
			\item Distinguishes red, yellow, green etc.
			\item More than 400 hues can be seen by the human eye.
		\end{itemize}
	\item Saturation (Degree of Dilution)
		\begin{itemize}
			\item Inverse of the quantity of ``white'' present in the signal.
			A pure colour has 100\% saturation, the white and grey have 0\% saturation.
			\item Distinguishes red from pink.
			\item About 20 saturation levels are visible per hue.
		\end{itemize}
	\item Brightness
		\begin{itemize}
			\item Amount of light emitted.
			\item Distinguishes the greylevels.
			\item The human eye perceives about 100 levels.
		\end{itemize}
\end{itemize}

\subsection{Munsel}
The Munsel colour system is a cylindrically mapped colour system.
\begin{itemize}
	\item Hue: 100 Equally spaced hues around circle
	\item Saturation: Starts at 0 on the centre line and increases to 10-18 depending on the hue
	\item Brightness: 0 for black up to 10 for white
\end{itemize}

\subsection{HSV}
The HSV colour system is the most intuitive model for users.
\begin{itemize}
	\item H = Hue (0.0\degree-360.0\degree)
	\item S = Saturation (Ratio from 0.0-1.0)
	\item V = Value (Intensity)(0.0-1.0)
\end{itemize}
The colour system is based on a hexcone shape.
When S=1 and V=1 ``akin to artists pure pigment'':
\begin{itemize}
	\item Hue: Change H
	\item Tints (Adding white): Decrease S
	\item Tone (Adding black): Decrease S and V
	\item Shade: S=1, decrease V
\end{itemize}

\subsection{HSL}
\begin{itemize}
	\item Hue
	\item Saturation
	\item Luminance
\end{itemize}
The HSL colour system uses a double cone representation.
Hue is represented 0\degree-360\degree.
Saturation is the value 0-1 out horizontally from the centre.
Luminance has a 0-1 value out vertically from the bottom.

\section{Colour Image Histograms}


\end{document}
