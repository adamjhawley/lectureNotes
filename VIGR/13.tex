\documentclass{article}

\usepackage{amsmath}
\usepackage{gensymb}

\author{Adam Hawley}
\title{Lecture 13: Drawing Lines \& Detecting Edges}

\begin{document}
\maketitle

\section{Pixels}
The pixel space is a rectangle on the $x,y$ plane, bounded by $0\le x\le x_{max}$ and $0\le y\le y_{max}$
Each axis is divided into an integer number of pixels, $N_x$ and $N_y$. The pixels therefore have width and height:
\begin{align*}
	W &= \frac{x_{max}}{N_x} \\
	H &= \frac{y_{max}}{N_y}
\end{align*}

Pixels are reffered to using integer coordinates, that either refer to the locatin of their left hand corners or their centres.
Knowing the $W$ and $H$ values allows the pixel to be defined.
Assuming that $W$ and $H$ are equal, the pixels are square.

\section{Line Rastersation}
Straight lines are described by the equation:
\begin{align*}
y = mx + c
\end{align*}

To draw a line we have to vary $x$ between a given range of values and compute a point $y$ at each location according to the line equation.
{\bf Rasterisation} is the process of converting the drawing to pixels in images (also called scan conversion).

\section{The Digital Differential Analyser Algorithm}
The Digital Differential Analyser Algorithm (DDA) is an algorithm used to do line rasterisation.
Assuming that we have two endpoints:
\begin{align*}
 p_0 &= (x_0,y_0) 
     &p_1 = (x_1,y_1)
\end{align*}
The point of intersection is given by the line rule where:
\begin{align*}
	m = \frac{y_1-y_0}{x_1-x_0}
\end{align*}
We can consider a point ($x,y$) on the line and then the next point is ($x+1,y+m$) when the line is drawn to the right.

The drawback of the DDA is that it uses folat and rounding operations for each point which require increased computational power.

\section{Bresenham's Algorithm}
The DDA algorithm needs to carry out a floating point addition and a rounding operaiton for every iteration.
Bresenham's algorithm operates only on integers requiriing only that the start ad end points of a line segment are rounded.

Firstly assuming that $0\le m \le 1$ and pixel ($i,j$) is the current pixel.
The value of $y$ on the line $x=i+1$ must lie between $j$ and $j+1$.
Because the gradient is in the range 0-$45\degree$ the next pixel must be either East ($i+1,j$) or North East ($i+1,j+1$).
We should also assume that pixels are identified by their centres.

At this point, we have a decision problem, and so must find a decision variable that can be tested to determine which pixel to colour.
First rewrite the equation of a straight line:
\begin{align*}
	y = mx + &c \\
	mx - y + c &= 0 \\
	\frac{\Delta y}{\Delta x} x - y + c &= 0 \\
	F(x,y) = x \Delta y - y\Delta x &+ B = 0
\end{align*}
The useful properties of this function are as follows:
\begin{itemize}
	\item $F(x,y)>0 \implies$ point is above the line
	\item $F(x,y)<0 \implies$ point is below the line
\end{itemize}

The function $F(x,y)$ clearly has the properties required of a decision variable.
Now all we have to do is apply it.
This will involve using rational numbers in the derivation even though we were supposed to be working with integers.

Consider an arbitrary pixel ($x_p,y_p$) that is on a line segment.
We need to test a point to determine which of the possible next 2 pixels to colour.
A suitable point is ($x_p+1,y_p+\frac{1}{2}$) this will be on the vertical boundary between pixels E and NE and at the horizontal midpoint.

\smallskip
If a $F(x_p+1,y_p + \frac{1}{2}) \le 0$ the next pixel is E.
Else, the next pixel is NE.
This is still under the same assumption that that $0\le m\le1$.

Examining the next step we see it is not independent of the decision we have just made.
\begin{itemize}
	\item If E was chosen then the next test will be $F(x_p+2,y_p+\frac{1}{2})$
	\item Else is NE was chosen then the next test will be $F(x_p + 2, y_p + \frac{3}{2})$.
\end{itemize}
Now let $d_n = F(x_p + 1, y + \frac{1}{2})$ and substitute the values $x+2, y+\frac{1}{2}$ and $y+\frac{3}{2}$ back into the definition of $F$ and we have:
\begin{itemize}
	\item If E was chosen then the next test will be $d_{n+1} = d_n+\Delta y$
	\item Else if NE was chosen the next test will be $d_{n+1} = d_n + (\Delta y - \Delta x)$
\end{itemize}

\end{document}
