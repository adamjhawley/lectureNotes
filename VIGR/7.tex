\documentclass{article}
\usepackage{amsmath}

\author{Adam Hawley}
\title{Lecture 7: Multiple View Vision}

\begin{document}
\maketitle

\section{Epipolar Geometry}
Considering two pinhole cameras with different projection centres, in order to relate the two images one can use {\it Epipolar Geometry}.

\begin{itemize}
	\item $p_l$ \& $p_r$ are the vectors from the centres of projections $O_l$ in the left and $O_r$ in the right image to the corresponding projections $P_l$ \& $P_r$ of the same 3D point $P$.
	\item $h_l$ \& $h_r$ are called {\bf epipolar lines} --- they are located at the interection between the image plane and the plane formed by the points P, $O_l$ \& $O_r$.
		Each point P will have an epipolar line in each image plane.
	\item $e_i$ \& $e_r$ are epipoles representing the intersection points of $O_l O_r$ with the left and right image planes.
		They may be located outside the actual images.
\end{itemize}


To estimate the epipolar geometry, determine the mapping between corresponding points in the two images.
\begin{align*}
	p_r &= Rp_l + t \\
	O_lO_r &= t
\end{align*}
The left and right images are connected by means of a matrix representing rotation {\bf R} in the plane $PO_lO_r$ and translation {\bf t}.  

\end{document}

