\documentclass{article}

\author{Adam Hawley}
\title{Lecture 12: Image Filtering}

\begin{document}
\maketitle
Recommended reading:
\begin{itemize}
	\item Forsyth \& Ponce Chapter 7
\end{itemize}

\section{Intoduction: Images as Functions}
An image can be seen as a function which is discretised and quantised to [0-255] levels.
%INSERT EXAMPLE
The function depends on the scene represented, light properties, image acquisition and so on.

\section{Image Subsampling}
Image subsampling consists of removing rows and columns from the image.

Noise is often lost through subsampling because noise is pixel values that are vastly different to their neighbours which do not represent small details.
These pixels are removed through subsampling.
This method is used commonly in deep learning and neural networks because large images require high levels of processing power and time.

When you zoom in on subsampled images, blurring is created, features related to noise and small details are lost.
This is because when you zoom in, pixels are duplicated to fill the new pixels which come from the increase in size.

\section{Image Interpolation}
\subsection{What is Image Interpolation?}
A better method for restoring the original image after subsampling is to use \textbf{image interpolation}.
\textbf{Image interpolation} uses neighbouring pixels to estimate the original pixel value between the neighbours.

\subsection{How is it done?}
The simples example would be to replicate a neighbouring pixel.
However, this is the same as zooming in on the image.
Instead it can be better to take an average of the neighbouring pixels.
\begin{itemize}
	\item \textbf{Bilinear interpolation} considers a weighted average of a 4x4 pixel neighbourhood.
	\item \textbf{Bicubic interpolation} considers a weighted average of an 8x8 pixel neighbourhood.
\end{itemize}

\end{document}
