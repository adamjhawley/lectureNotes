\documentclass{article}

\usepackage{amsmath}

\newcommand{\ninth}{\frac{1}{9}}

\author{Adam Hawley}
\title{Lecture 12: Image Filtering}

\begin{document}
\maketitle
Recommended reading:
\begin{itemize}
	\item Forsyth \& Ponce Chapter 7
\end{itemize}

\section{Intoduction: Images as Functions}
An image can be seen as a function which is discretised and quantised to [0-255] levels.
%INSERT EXAMPLE
The function depends on the scene represented, light properties, image acquisition and so on.

\section{Image Subsampling}
Image subsampling consists of removing rows and columns from the image.

Noise is often lost through subsampling because noise is pixel values that are vastly different to their neighbours which do not represent small details.
These pixels are removed through subsampling.
This method is used commonly in deep learning and neural networks because large images require high levels of processing power and time.

When you zoom in on subsampled images, blurring is created, features related to noise and small details are lost.
This is because when you zoom in, pixels are duplicated to fill the new pixels which come from the increase in size.

\section{Image Interpolation}
\subsection{What is Image Interpolation?}
A better method for restoring the original image after subsampling is to use \textbf{image interpolation}.
\textbf{Image interpolation} uses neighbouring pixels to estimate the original pixel value between the neighbours.

\subsection{How is it done?}
The simples example would be to replicate a neighbouring pixel.
However, this is the same as zooming in on the image.
Instead it can be better to take an average of the neighbouring pixels.
\begin{itemize}
	\item \textbf{Bilinear interpolation} considers a weighted average of a 4x4 pixel neighbourhood.
	\item \textbf{Bicubic interpolation} considers a weighted average of an 8x8 pixel neighbourhood. 
	\item \textbf{Adaptive interpolation} depends on image context.
\end{itemize}

\section{Image Convolution}
\textbf{Image convulation} uses a window of a neighbourhood.
Kernel $w(x,y)$ with values for each window element.
Calculates the output using the following formula:
\begin{align*}
	g(x,y) = \sum^K_{s=-K}\sum^K_{t=-K}w(s,t)f(x+s,y+t)
\end{align*}
Applying this formula assigns the output to the central element of the $2K+1 \times 2K+1$ window.
Afterwards, the centre moves to the next neighbourhood according to the following:
\begin{align*}
	s &= 1,..., image\_width-1 \\
	t &= 1,..., image\_height-1
\end{align*}

\section{Applications of Image Convolution}
\begin{itemize}
	\item Filtering for removing noise
	\item Sharpening images and image features such as object edges and texture
	\item Producing graphical artistic effects in images for \textit{softening} and \textit{sharpening} images, ``fire'' effects and embossing images.
	\item Colouring images as vectors.
\end{itemize}

\section{Image Filtering}
Image filtering can be usually applied simply by convolution with a filter mask.

\subsection{The Mean (Averaging) Filter}
Consider the convolution of an image with the follwing kernal for a $3 \times 3$ window:
\smallskip
\centerline{$
\begin{bmatrix}
	\ninth & \ninth & \ninth \\
	\ninth & \ninth & \ninth \\
	\ninth & \ninth & \ninth
\end{bmatrix}
$}
The output is the average of the 9 pixels.
This results in the removal of noise but also blurring of the image.

\subsection{Gaussian Filter}
The Gaussian filter represents a weighted averaging filter, where the weights are higher in the central region of the window according to the Gaussian equation:
\begin{align*}
G(x,y) = \frac{1}{2\pi\sigma^2}e^{-\frac{x^2+y^2}{2\sigma^2}}
\end{align*}
Note: in graphics, $G$ is discretly approximated. 

\section{Image Noise}
Noise occurs naturally in images due to the camera sensors --- increase camera sensitivity (ISO) --- increase noise.
Noise can be removed by filtering.

\medskip
\begin{align*}
	\hat{f}(x,y) = f(x,y) + n(x,y)
\end{align*}
Where $\hat{f}$ is the noisy image, $f$ is the original image and $n$ is the noise itselve.
We can test various filters by artificially adding noise to the image and filtering it afterwards.
\subsection{The Mean  Filter}
Images are invariably noisy --- for example when increasing ISO we achieve better detail and better night images --- but also it adds noise to images.

Averaging corresponds to a low pass filter and removes some of the noise by smoothing.
On the other hand averaging filters lead to blurring and can remove small details.
It is also very effective in temporal image sequences.

Problems with the mean filter:
\begin{itemize}
	\item Blurs edges and details in the image
	\item Not effective for impule noise (\textit{Salt-and-Pepper})
\end{itemize}

\textbf{\textit{Salt-and-Pepper Noise}}:\\
Add extreme values to the image of black (0) and white (255) according to a certain percentage.

\subsection{Median Filter}
The \textbf{median filter} ranks the data inside the window and picks the middle value.

Note: since the median filter simply takes the middle value of the window, it does not need to be a square image.

\section{Rank Order Statistics Filters}
Rank order statistics filters rank the values from the window and pick the value of a certain rank.
The most common rank order statistics filters used are minimum and maximum filters --- these can be used for segmenting the image. 
(See section \ref{maxmin})

\section{Maximum \& Minimum Filters} \label{maxmin}
\subsection{Maximum Filter}
The max filter takes the maximum value in the neighbourhood.
This has the effect of enhancing bright areas.

\subsection{Minimum Filter}
The min filter takes the minimum value in the neighbourhood.
This has the effect of introducing darkness in the image.

\section{Conclusions}
What you need to know:
\begin{itemize}
	\item Image interpolation
	\item Image convolution
	\item Image neighbourhoods
	\item Mean filter
	\item Median filter
	\item Gaussian and salt-and-pepper noise
	\item Min and max filters
\end{itemize}



\end{document}
