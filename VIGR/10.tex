\documentclass{article}

\usepackage{amsmath}

\title{Lecture 10: Light \& Reflectance}
\author{Adam Hawley}

\begin{document}

\maketitle

\tableofcontents
\newpage

\section{Light and Reflectance}
Images are produced by light falling on a camera sensor:
\begin{itemize}
	\item The light must come from somewhere
	\item It may be reflected from surfaces before it reaches the camera
\end{itemize}

Hence we need models of:
\begin{itemize}
	\item Light sources
	\item Reflectance of light from surfaces
\end{itemize}

Illuminating a surface gives rise to shading --- very important for realism and the perception of 3D shape.

\section{Importance of Shading}

We perceive curved surfaces through their shading patterns.
Shading arises when we add a light source to our scene and employ a reflectance model.

\section{Image Formation Process}
Factors that affect the amount of light reflected towards us:
\begin{itemize}
	\item Object shape and in particular the angle of the surface; if the surface is perpendicular to the light from the source it is more likely to reflect light.
	\item Different materials; some materials can give very different reflections of light such as a snooker ball which gives one small concentration of bright reflection compared with velvet which gives a much more even amount of light reflected in a bigger region.
	\item Illumination conditions
	\item Viewpoint/direction, camera parameters.
\end{itemize}

The surface ``reflectance properties'' of an object are described by a {\bf reflectance} {\bf model} (or illumination model). 

\section{Distant Vs. Local Illumination}
\subsection{Distant Illumination}
Light sources which are very far away (e.g the sun) are often assumed to be at an infinite distance.
This allows for two simplifications:
\begin{itemize}
	\item Direct constant within scene
	\item No attenuation
\end{itemize}
\subsection{Local Illumination}
Light sources situated within the scene (local) are more complex:
\begin{itemize}
	\item Direction varies within the scene
	\item Objects closer to the source are illuminated more strongly than those futher away (there is greater {\textbf {\textit {light source attenuation}}})
\end{itemize}

\section{Light Source Attenuation}
The physically correct model for attenuation of a light source follows the classical one over distance squared form:
\begin{align*}
	f_{att} = \frac{1}{d^2_L}
\end{align*}
Where $d_L$ is the distance to the light source.

Even though this is physically correct, this sometimes gives too severe a fall off to be used in graphics.
For this reason, alternative functions are sometimes used.
For example:
\begin{align*}
	f_{att} = \frac{1}{d_L}
\end{align*}

\section{Point Sources and Spotlights}
The simplest light source type is a {\bf point source}.
Characterised by a 3D location (local) or 3D direction vector (distant) and an intensity or colour.
Sometimes the intensity of a distant light source is encoded in the length of the direction vector.

A {\bf spotlight} is a point source that only emits light in a constrained cone of angles.
For example: studio ``barn door'' type spotlights.

\section{Ambient \& Distributed Sources}
{\bf Ambient light} is light that arrives equally from all directions.
This makes it a good model for a cloudy day.
It also leads to very {\it flat} images which lack shading.

{\bf Distributed sources} emit light from an area.
This means that regions may be illuminated by part of the source.
It leads to soft shadows: penumbra around shadowed regions.

\section{Environment Maps}
In reality, light arrives at a point from all directions.
One useful approximation iss a spherical {\bf environment map}.
Each pixel is like a distant point source with its own direction and colour.

%CHECK LECTURE TO SEE WHAT WE NEED TO KNOW ABOUT THE RENDERING EQUATION
\section {The Reflectance Equation}
See slides.

\section{Local Vs. Global Reflectance}
In practice, light bounces off surfaces onto other surfaces (perhaps many times) before reaching our eye.
If it didn't, then any surface  not directly illuminated would be invisible.
Light reflecting from one surface to another is called {\bf interreflection}.
In this lecture, we consider only {\bf local models of reflectance}.
In this case, we treat each point separately and only consider illumination coming directly from a light source.

\section{Reflectance Models}
Designing realistic reflectance models is a large part of current computer graphics research.
Many materials have complex reflectance properties:
\begin{itemize}
	\item Human skin
	\item Fur
	\item Velvet
	\item Hair
	\item Brushed metals
	\item The underside of a CD!
\end{itemize}
A reflectance model describes what happens when light from a certain direction strikes a surface.
When this happens light may be:
\begin{itemize}
	\item Absorbed (As on a \textbf{Translucent Surface})
	\item Scattered (As on a \textbf{Diffuse Surface})
	\item Reflected (As on a \textbf{Specular Surface})
	\item Transmitted into the surface
\end{itemize}
Or any combination of the above.

Almost all surface reflectance models are of the form:
\begin{align*}
	I = f(n,s,v,P)
\end{align*}
Where $n,s,v$ are the three unit vectors describing direction: \textbf{n} - surface normal, \textbf{s} - light source \& \textbf{v} - viewer.
$P$ generalises some other parmeters describing the material properties of the surface.

\subsection{Ambient Reflection Model}
The simplest reflectance model assumes completely ambient illumination:
\begin{align*}
I = L_ak_a
\end{align*}
Where $L_a$ is the strength of the light source and $k_a$ is the ambient coefficient ($0<k_a\leq1).$
$k_a$ is our first example of a material property.
In genereal, we will have a value for each of the RGB channels.

This simple ambient model is not realistic.
It is generally added to models because graphics programmers noticed their models looked too dark otherwise and the shading drop off at the edge of objects was too drastic.
It ignores the fact that parts of an object may occlude others, reducing the amount of ambient light that reaches the surface.

(We will see a better model for ambient reflectance later)

\subsection{The Lambertian Reflection Model}
The simples model which accounts for changes in intensity due to changes in surfaqce orientatin is the \textbf{Lambertian Model}.
This is based on the idea of a \textbf{perfecttly diffuse reflector}.
The assumption is that if the surface is macroscopically rough, light will be scattered equally in all directions.
Similarly, it may be that the light which enters the surface is scattered internally and re-emerges completely diffused.
This model is good for materials like paper and chalk.

Where the angle between surface normal and light source, $\theta_i$, is the \textbf{angle of incidence}, the Lambertian intensity is simply:
\begin{align*}
	I = L_dk_d\cos(\theta_i)
\end{align*}
Or in terms of vectors:
\begin{align*}
	I = L_dk_d(\textbf{n}\cdot\textbf{s})
\end{align*}
Note: there is no \textbf{v} since diffuse reflection is independant of viewing direction.
In general, we assume that the light source is distant and hence that $L$ is constant over the entire image.

\subsection{Specular Reflection \romannumeral1: The Phong Model}
The Lambertian model gives a dull, matte appearance.
Many real surfaces are macroscopically smooth and appear shiny.
Reflection from these sorts of surfaces is known as specular reflectance.
A perfect specualor reflector only reflects light in the direction \textbf{r}.
Imperfect specular reflectors reflect light into a range of directions around \textbf{r}.
So strength of specular reflectance is determined by andle \textbf{r} and \textbf{v}.
\textbf{r} is the reflection of \textbf{s} about \textbf{n}.
So \textbf{r}, \textbf{s} and \textbf{n} lie in the same plane.
\textbf{r} can be found from \textbf{s} and \textbf{n}: 
\begin{align*}
	\textbf{r} = 2(\textbf{n}\cdot\textbf{s})\textbf{n})-\textbf{s}
\end{align*}
The Phong specular model is a simple phenomenological model of specular reflectance for non-perfect reflectors.
The maximum reflectance is observed when $\theta_{spec} = 0$ where $\theta_{spec}$ is the angle between \textbf{r} and \textbf{v}.
Rapid fall-off is determined by the \textbf{shininess coefficient}, $\eta$.
The specular term is therefore given by:
\begin{align*}
	I &= L_sk_s\cos^{\eta}\theta_{spec} \\
	I &= L_sk_s(\textbf{r}\cdot\textbf{v})^\eta
\end{align*}

\subsection{Specular Reflection \romannumeral2: The Blinn-Phong Model}
A problem with computing the perfect reflection direction, \textbf{r}, is that \textbf{n} is different for every point on the surface, so we must recompute \textbf{r} for every polygon --- very slow.
There is a common simplification known as the \textbf{\textit{Blinn-Phong Model}}.
Instead of \textbf{r}, use aa vector halfway between the viewer and the light source.
Since we assume \textbf{s} is distant, it is fixed over the image so we only need to calculate \textbf{h} once per image:
\begin{align*}
	\textbf{h} = \frac{\textbf{s} + \textbf{v}}{\|\textbf{s} + \textbf{v}\|}
\end{align*}
If \textbf{n}=\textbf{h}, we have perfect specular direction.

\subsection{The Complete Phong Model}
The full Phong model is simply the sum of the ambient, diffuse and specular terms, summing over all lights:
\begin{align*}
	I = k_aL_a + \sum_{lights}(k_d(\textbf{n$\cdot$s})L_d + k_s(\textbf{n$\cdot$h)}^{\eta}L_s)
\end{align*}

\section{Colour in Reflectance Models}
In all of the reflectance model equations given so far, material properties (ambient, diffuse, specular coefficients and shininess) are \textbf{wavelength dependent}.
The strength of the light source is also wavelength dependent (i.e light source has a colour).
Some of the latest reflectance models evaluate intensity over the whole visible spectrum.
In practice, we provide RGB values for light sources and ambient, diffuse and specular coefficients.
Often:
\begin{itemize}
	\item Ambient RGB = Diffuse RGB (Object Colour)
	\item Specular RGB = White
\end{itemize}

\section{Summary}
What you should know:
\begin{itemize}
	\item Types of light source
	\item The rendering equation
	\item Local reflectance models
	\item The Phong model: Ambient, diffuse and specular reflectance
\end{itemize}

\end{document}
