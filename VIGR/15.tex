\documentclass{article}

\usepackage{amsmath}

\author{Adam Hawley}
\title{Lecture 15: Global Illumination \& Image-Based Lighting}

\begin{document}

\maketitle
\tableofcontents
\newpage

\section{Introduction}
Up until this lecture, we have only been modelling illumination in purely local terms.
Reflected intensity is a function solely of light arriving at a point directly from light sources.
\textbf{Global illumination} aims to model all illumination which has arrived at a point including light which has been reflected or refracted by other objects.

\section{Global Interactions}
The general approach to considering global interactions:
\begin{itemize}
	\item Start with the light source and follow every ray of light as it travels through the environment.
\end{itemize}
Stopping under any of the following conditions:
\begin{itemize}
	\item Light hits the eye point
	\item Light travels out of the environment
	\item Light has its energy reduced under an admissable limit due to absorption in objects.
\end{itemize}
This process is known as \textbf{\textit{photon mapping}}.
This is computationally very expensive
Certainly not practical for real time graphics but is used offline.

\section{Types of Interactions}
Diffusely radiated light reflected off another diffuse surface is called \textbf{Radiosity}.
Specular radiated light reflected off another specular surface is called \textbf{Ray Tracing}.

\section{Ray Tracing}
\subsection{Whitted Ray Tracing}
The following outlines Whitted ray tracing:
\begin{itemize}
	\item Traces light rays in the reverse direction of propogation from the eye back into the scene towards the light source.
	\item Spawn reflection rays and refraction rays for every hit point (recursive calculation).
\end{itemize}
Global illumination model (specular) and a local model.
Considers diffuse-reflection interactions but not diffuse-diffuse interactions.

Whitted's illumination equation:
\begin{align*}
	I = k_aL_a + \sum^m_{i=1}f_{att_i}L_i[(k_d(\textbf{n}\cdot\textbf{s}_i)+k_s(\textbf{n}\cdot\textbf{h}_i)^{\eta})] + k_rI_r + k_tI_t
\end{align*}
This equation consists of the standard Blinn-Phong model combined with two extra terms; describing reflected light and transmitted light.
These two additional terms contain the recursive elements.

%CHECK LECTURE SLIDES FOR RAY TRACING PSEUDOCODE 

\subsection{Problems with Ray Tracing}
Ray tracing, in its current state has several disadvantages:
\begin{itemize}
	\item Shadow rays usually not refracted
	\item Numerical precision --- false shadows
	\item Simplistic model of ambient light
	\item Expensive to compute --- lots of ray-surface intersection computations
\end{itemize}
Note: issues with numerical precision can be alleviated using multiple rays per pixel or sub-pixel jitter --- gives soft shadows.

\section{Image-Based Illumination}
Ray tracing allows us to more realistically model how light is transferred around a scene.
This gives a big improvement in realism.
But modelling of the scene (objects+illumination) is a very laborious process.
Humans perceive material properties and shape better under \textbf{natural illumination}.
The easiest way to measure natural illumination is to take a photograph using a mirrored ball (light probe).
The mirrored ball measures incident illuminatin from every direction.
The next step is to resample the picture to ``equirectangular mapping''.
This is called an \textbf{environment map}.
The mapping has two dimensions: elevation ($y$) and azimuth ($\sim x$)

To convert ($x,y$) coordinates to angles:
\begin{align*}
\theta(x) &= \pi\left ( \frac{2x}{x_{max}}-1\right ) \\
	\phi(y) &= \frac{y\pi}{y_{max}}
\end{align*}
To convert angles to a direction (unit vector):
\begin{align*}
	\textbf{s}(x,y) = 
	\begin{bmatrix}
		\sin(\phi(y))\sin(\phi(x)) \\
		\cos(\phi(y)) \\
		\sin(\phi(y)) \cos(\phi(x))
	\end{bmatrix}
\end{align*}
Each pixel (x,y) in an environment map $I(x,y)$ corresponds to a light source with direction \textbf{s}(x,y) and colour $I(x,y)$
To render with an environment map, we just sum over light sources (every pixel in environment map image!) as in Lecture 10.

\section{Environment/Reflection Mapping}
Very cheap approximation to appearance of mirror specular object under environment illumination.
Reflect viewing direction about surface normal giving specular lighting direction:
\begin{align*}
	\textbf{s} = 2(\textbf{n}\cdot \textbf{v})\textbf{n}-\textbf{v}
\end{align*}

Use this to look up colour in environment map and copy to pixel.
Environment map stored as texture on reference object such as sphere or cube.
Possible in real time at very low computational cost.

\section{Occlusions}
A point on a surface will not necessarily be able to see all of the environment.
i.e Point may be shadowed from certain sources.
Half will be excluded because of self shadowing.
Only those in ``upper hemisphere'' considered.

To handle occlusions exactly, every direction from every surface point needs checking --- like ray tracing with thousands of light sources.

Alternatively: \textbf{precompute} proportion of hemisphere that is occluded and use with environent lighting.
This is \textbf{ambient occlusion}.

\subsection{Ambient Occlusion}
\begin{itemize}
	\item Ambient occlusion = 1: Completely unoccluded --- use all sources in upper hemishphere.
	\item Ambient occlusion = 0: Completely occluded --- no light reaches point
\end{itemize}
The \textbf{bent normal} is the average unoccluded direction.

Ambientt occlusion is exact for an environment of perfectly ambient light.
It gives rise to a different kind of shading that is important in human perception.

To use ambient occlusion with general environment lighting:
\begin{enumerate}
	\item Render surface while ignoring occlusions
	\item Downweight resulting intensities by ambient occlusion amount
	\item Use bent normal instead of surface normal
\end{enumerate}
The bent normal is used to describe the dominant direction from which light arrives.
It can be unrepresentative when there are ``\textit{blockers}'' in the scene.

\section{Photon Mapping}
Photon mapping approximates the full rendering equation.
Considers all types of interaction and can support image-based lighting.
Based on the \textit{Monte Carlo} simulation:
\begin{enumerate}
	\item Fire photon from light source.
	\item When photon strikes a surface, randomly select a behaviour according to a probabilistic model (e.g. may be absorbed, may be reflected).
	\item Count number of photons that strike each pixel.
\end{enumerate}
The more photons we fire, the better the approximation.
Can include arbitrarily complex models of photon behaviour when striking a surface (e.g. subsurface scattering through multiple layers of a surface).
Can incluse wavelength-dependent effects.

\section{Conclusions}
What you should know:
\begin{itemize}
	\item What is global illumination
	\item Recursive ray tracing
	\item Image-based lighting
	\item Ambient occlusion
\end{itemize}

\end{document}
