\documentclass{article}
\usepackage{amsmath}
\usepackage{graphicx}
\graphicspath{{./images/}}

\author{Adam Hawley}
\title{Lecture 6: Projections}

\begin{document}

\maketitle

\section{Viewing Systems}
In real life we pick up objects, position them and then view them.
In computer graphics objects are positioned in a fixed frame.
The viewer moves to the appropriate position in order to achieve the desired view.

\section{Perspective Projections}
We assume that there exists a {\it Centre of Projection} (COP).
Objects' distances from the {\bf COP} cause the objects to appear larger or smaller on the intersecting viewplane. %EXAMPLE 

Lines that are not parallel to the viewplane converge to a vanishing point. 
The {\it Principal Vanishing Point} exists for lines that are parallel to the principal axis.  
This can be clearly seen in the perspective projection of a cube, edges which are parallel to one another appear to converge. %NEEDS EXAMPLE 

\section{Parallel/Orthographic Projections}
Here the COP is {\bf always} at infinity.
This means that the viewplane is aligned with the axes and the {\it Direction of Projection} (DOP)
\end{document}
