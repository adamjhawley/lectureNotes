\documentclass{article}
\usepackage{amsmath}

\author{Adam Hawley}
\title{Lecture 8: Colour Perception}

\begin{document}

\maketitle

\section{The Light Spectrum}
\subsection{Visible Light}
Radiation in nature is characterised by wavelength and frequency.
Visible light has a very narrow wavelength range of 350nm - 780nm (neighbouring to ultraviolet and infrared respectively).

\subsection{Visualising Other Spectra}
In order to visualise other spectra, we can use the appropriate sensors and lenses.
For example, visualising infrared radiation has many uses including security and medicine.

Most of the time, we use {\it true colour} to represent radiation as it is what we use in photographs.
However, when visualising other spectra which are not within the reach of true colour, one can use {\it pseudocolour} or {\it greylevel mapping} to represent the radiation.
\begin{itemize}
	\item {\bf Pseudocolour} is when the image uses colours re-assigned according to an LUT or function.
	\item Similarly, {\bf greylevel mapping} represents the radiation using a function however the product of this function represents a grey level value and no colours are involved.
\end{itemize}

\section{Hyperspectral Images}
\subsection{What Are Hyperspectral Images?}
A {\it hyperspectral image} associates a vector of various wavelengths to a pixel.
This results in volumetric images where one of the dimensions is the wavelength (spectral dimension).
The information in each pixel depends on the capacity of the corresponding object surface to absorb and reflect radiation of that specific wavelength.

Hyperspectral imaging collects and processes information from a large range of wavelengths using specific sensors. %INCLUDE GRAPHIC

\subsection{Appplications}
Hyperspectral imaging can be used for a number of things:
\begin{itemize}
	\item Identification of materials by measuring their spectral absorption and reflectance.
	\item Discovery of natural resources by using ground-penetrating images taken from aeroplanes and satellites.
	\item Discovery of concealed objects
	\item Assessing the quality of fruit and vegetables
\end{itemize}

\section{Illumination}
Light sources are characterised by their own spectral range $[\lambda_1-\lambda_2]$.
The spectral range interacts with scene colours which means that the output in the image depends on the light spectral range of the light source.
For example, consider the difference between some photos with and without a flash or at different times of day.

\section{CIE Colour Model}
\subsection{Outlining the CIE}
{\bf Commision Internatinal de l'Eclairage} (CIE).
Three standard primaries $x,y,z$ for representing all colours perceived by humans.

Where {\bf C} is a colour to be matched: 
\begin{align*}
	{\bf C} = X.{\bf X} + Y.{\bf Y} + Z.{\bf Z} 
\end{align*}
This can be thought of as a 3D space.
Chromacity values depend on dominant wavelength and saturation (not the luminance) through normalising ($X,Y,Z$):
\begin{align*}
	x &= \frac{X}{X+Y+Z} & y &= \frac{Y}{X+Y+Z} & z &= \frac{Z}{X+Y+Z}
\end{align*}
Since $x + y + z = 1$ we only need ($x,y$) to specify the chromactiy value, if we want the luminance information then we just need ($x,y$) and $Y$ since:
\begin{align*}
	X &= \frac{x}{y}Y & Z &= \frac{1-x-y}{y}Y
\end{align*}

Aspects of the chromacity diagram:
\begin{itemize}
	\item Interior corresponds to visible colours
	\item Pure colours appear along the curve
	\item {\bf Illuminant C} is considered to be `standard daylight'  
	\item Luminance dependent colours do not appear (e.g brown) 
\end{itemize}

\subsection{Selecting Colours Using the CIE}
The diagram allows us to determine the dominant wavelength and {\it excitiation purity} of light:
\begin{enumerate}
	\item Match colour A using the 3 CIE primaries
	\item Plot the position in the CIE chromacity diagram
	\item The mix of two colours lie on a straight line between them
\end{enumerate}

Complementay are those that can be mixed to yield white light.
Colours which have no dominant wave-length are referred to as non-spectral.

\end{document}
