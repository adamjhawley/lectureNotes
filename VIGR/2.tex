\documentclass{article}
\usepackage{amsmath}

\author{Adam Hawley}
\title{Lecture 2: Transformations}

\begin{document}

\maketitle
{\bf References:} Nielsen 3.2.1 \& 3.2.2 
\tableofcontents
\newpage

\section{Conventions}
\begin{itemize}
	\item Matrices written in uppercase bold eg:
{\bf A} =$ 
\begin{bmatrix}
1&2\\
3&4
\end{bmatrix}
$
\item Matrices are also indexed (row, column) from top left starting at 1.
\item Individual elements are written as lowercase: 
	\\  \centerline{ $ a_{11} = 1, a_{12} = 2, a_{21} = 3, a_{22} = 4 $ }
\item Vectors are simply matrices with a single row or column represented by a lower-case bold symbol:

	\centerline{ $ {\bf a} = 
		\begin{bmatrix}
			x\\y\\z
		\end{bmatrix} OR \ \ {\bf a} = 
		\begin{bmatrix}
			x&y&z
		\end{bmatrix}
	$ }
\end{itemize}

\section{Transformations}
\subsection{Translation}
Translation involves {\textbf{\textit{component-wise addition}}}.

For example: \\
\centerline{$
\begin{bmatrix}
	x' \\ y'
\end{bmatrix}
=
\begin{bmatrix}
	x \\ y
\end{bmatrix}
+
\begin{bmatrix}
	d_x \\ d_y
\end{bmatrix}
$}
\subsection{Scaling}
Scaling involves {\textbf{\textit{multiplication by a scalar}}}.

For example: \\~\\
\centerline{$
\begin{bmatrix}
x' \\ y'
\end{bmatrix}
= s 
\begin{bmatrix}
x \\ y
\end{bmatrix}$}

Or to have a separate dimensional scalars:\\ \\
\centerline{$
\begin{bmatrix}
x' \\ y'
\end{bmatrix} = 
\begin{bmatrix}
	s_x&0\\
	0&s_y
\end{bmatrix} $}
\subsection{Rotation}
Rotation involves {\textbf{\textit{matrix multiplication}}}. Using the formula below to give a 2D rotation counter clockwise by angle $\theta$:
\\~\\
\centerline{$
	\begin{bmatrix}
		x' \\ y'
	\end{bmatrix} = 
	\begin{bmatrix}
		\cos(\theta) & -\sin(\theta)\\
		\sin(\theta) & \cos(\theta)
	\end{bmatrix}
	\begin{bmatrix}
		x \\ y
	\end{bmatrix}
$}

Rotation preserves distances: \\
\centerline{$ ||{\bf x}|| = ||{\bf Rx}|| $}\\
\centerline{$||{\bf x}||^2 = x^2_1 + x^2_2 + ... + x^2_n = {\bf x}{\it ^T}{\bf x} $}
\centerline{${\bf x}{\it ^T}{\bf x} = ({\bf Rx}){\it ^T}({\bf Rx}) 
$}
\centerline{${\bf x}{\it ^T}{\bf Ix} = ({\bf x}){\it ^T}({\bf R}{\it ^T})({\bf Rx}) 	
$}
\centerline{$ = {\bf x}{\it ^T}({\bf R}{\it ^T}{\bf R}){\bf x}  
$}
And hence: \\ \centerline{$ {\bf R}{\it ^T}{\bf R} = {\bf I} $} 
\centerline{$ {\bf R}{\it ^T} {\bf R} {\bf R}^{-1} = {\bf IR}^{-1} $}    
\centerline{$ {\bf R}{\it ^T} = {\bf R}^{-1}  $}
So to get the inverse of a rotation you only need to transpose it.
\subsection{Affine Transformations}
All of the previously mentioned transformations have been {\bf affine} transformations. These are transformations which can be expressed as matric multiplication and addition: 

\centerline{$ \bf y = Ax + b $} 

For 2D, {\bf A} is a 2x2 matrix and {\bf b} a 2x1 vector.

For 3D, {\bf A} is a 3x3 matrix and {\bf b} a 3x1 vector.  

\section{Homogeneous Coordinates}
\subsection{Defining Homogenous Coordinates}
Representing the affine transformations can be done by moving from Cartesian to homogeneous coordinates.

All 2D affine transformations can be then represented by a 3x3 matrix (or 4x4 for 3D)

\centerline{$
	\begin{bmatrix}
		x \\ y
	\end{bmatrix}
	becomes
	\begin{bmatrix}
		x \\ y \\ 1
	\end{bmatrix}
$}

\subsection{Affine Transformations as Homogeneous Coordinates}
The extra dimension allows uniform treatment of transformations:

Translation: $\begin{bmatrix}x' \\ y' \\ 1 \end{bmatrix}
= \begin{bmatrix}1&0&d_x\\0&1&d_y\\0&0&1\end{bmatrix}
\begin{bmatrix} x\\y\\1\end{bmatrix}$ 

Scaling: $\begin{bmatrix}x'\\y'\\1\end{bmatrix}
= \begin{bmatrix}s_x&0&0\\0&0&s_y\\0&0&1\end{bmatrix}
\begin{bmatrix}x\\y\\1\end{bmatrix}$

Rotation: $\begin{bmatrix}x'\\y'\\1\end{bmatrix}
=\begin{bmatrix}\cos\theta&-\sin\theta&0\\
		\sin\theta&\cos\theta&0\\
		0&0&1\end{bmatrix}
\begin{bmatrix} x\\y\\1	\end{bmatrix}$\\
\vspace{2.5mm}
\vspace{2.5mm}
\\
In general, any affine transformation:

\centerline{$\bf y = Ax + b$}

Can be expressed as:

\centerline{$ \begin{bmatrix}y\\1\end{bmatrix}=
	 \left[ \begin{array}{ccc|c}
	 &{\bf A}&& {\bf b} \\ 0&...&0&1
	\end{array} \right] 
	\begin{bmatrix}
	{\bf x}\\1
	\end{bmatrix}
$}

{\bf Note:} Two homogeneous points are equal if they differ only by a scale factor.
\end{document}
