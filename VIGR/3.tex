\documentclass{article}
\begin{document}
\author{Hawley, Adam}
\title{Lecture 3: Photometric Image Formation\\Part 1}
\maketitle
\section{Types of Sensor}
There are many different types of sensors used in computer vision:
\begin{itemize}
	\item Optical
	\begin{itemize} \item CCD, Photodiodes, Photomultipliers \end{itemize}
	\item Infra-red (thermal imaging cameras)
	\begin{itemize} \item CCD (Cooled), Photodiodes \end{itemize}
	\item Synthetic Aperture Radaaar (SAR)
	\begin{itemize} \item Radar, Antenna \end{itemize}
	\item Range Sensors
	\begin{itemize} \item Laser \& Photodiode \end{itemize}
	\item MRI
	\begin{itemize} \item Magnetic field gradients applied causing production of rotating magnetic field which can be measured. \end{itemize}
	\item PET/CAT
	\begin{itemize} \item Simulated radiation emission via magnetic field or radio isotope. \end{itemize}
\end{itemize}
\section{From Light to Images}
\subsection{Definitions}
\begin{itemize}
	\item {\textit{\textbf {Irradiance}}}: power incident on a surface (power per unit area).
	\item {\textit{\textbf {Radiance}}}: power travelling from a source (power per unit solid angle per unit projected source area).
\end{itemize}
\end{document}
