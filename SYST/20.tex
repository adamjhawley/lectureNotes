% Created 2019-04-08 Mon 12:46
% Intended LaTeX compiler: pdflatex
\documentclass[11pt]{article}
\usepackage[utf8]{inputenc}
\usepackage[T1]{fontenc}
\usepackage{graphicx}
\usepackage{grffile}
\usepackage{longtable}
\usepackage{wrapfig}
\usepackage{rotating}
\usepackage[normalem]{ulem}
\usepackage{amsmath}
\usepackage{textcomp}
\usepackage{amssymb}
\usepackage{capt-of}
\usepackage{hyperref}
\author{Adam Hawley}
\date{\today}
\title{Lecture 20: Network Layer \& Internet Protocol Continued}
\hypersetup{
 pdfauthor={Adam Hawley},
 pdftitle={Lecture 20: Network Layer \& Internet Protocol Continued},
 pdfkeywords={},
 pdfsubject={},
 pdfcreator={Emacs 26.1 (Org mode 9.2)}, 
 pdflang={English}}
\begin{document}

\maketitle
\tableofcontents


\section{Intra-Domain Routing}
\label{sec:orgc3bc4c0}
Routing is needed for forwarding packets in a datagram (connectionless) network, or for establishing virtual circuits in a VC (connection oriented) network.
Routing algorithms or protocols create routing tables from which one may derive the neccesary forwarding tables.
These in turn, define the output port through which a packet will be forwarded.

Most routing protocols work only for 10s or 100s of nodes and hence they are referred to as \textbf{interior gateway protocols (IGPs)} or (\textbf{intra-domain routing protocols}).
To make them scale, internetworks employ a hierarchical routing structure based on \textbf{domains}.
\begin{itemize}
\item A \textbf{domain} is an internetwork where all routers are under a single administritative entity (e.g. university campus).
\item Each domain uses IGPs to route packages within its boundaries and uses gateway routes to forward packets to other domains (inter-domain routing).
\end{itemize}

\subsection{Graph Representation of Routing}
\label{sec:org055c70c}
Routing is a graph-theoretic problem and requires one to calculate the lowest-cost path between two nodes.
\begin{itemize}
\item Nodes are hosts, switches, routers or networks
\item Edges are network links, each associated with a cost.
\item Cost of a path is the sum of the costs of all traversed edges.
\end{itemize}
There are two main types of algorithms for solving the problem:
\begin{itemize}
\item Global routing: all routers have complete topology and link cost info --- "link state" algorithms.
\item Decentralised routing: router knows link costs to neighbours --- "distance vector" algorithms.
\end{itemize}

\subsection{Link State Routing Algorithm}
\label{sec:orgc9863c3}
Dijkstra's Shortest Path Algorithm is based on a link state broadcast, aiming to provide all nodes with the same information.
After k iterations, the algorithm knows the least cost path to k destinations.
Notation is as follows:
\begin{description}
\item[{c(x,y)}] Link cost from node x to y, c = \(\infty\) if not direct neighbours
\item[{D(V)}] Current value of cost of path from source to destination v.
\item[{p(v)}] Predecessor node along path from source to v.
\item[{N'}] Set of nodes whose least cost path is definitively known.
\end{description}

\subsection{Distance-Vector Routing Algorithm}
\label{sec:orgeede7e1}
One classic example is the \textbf{Bellman-Ford routing algorithm} (1957) and \textbf{Ford-Fulkerson algorithm} (1962).
It is a \textbf{dynamic routing algorithm} and was used in the Internet Protocol under the name RIP (Routing Information Protocol).

Each node stores an array (a \textbf{vector}) containing the currently believed distances to all other nodes.
Initially this distance is \(\infty\) for all nodes except the considered node's immediate neighbours.
Vectors are distributed to a node's immediate neigbours.




\subsection{Dimension-Order Routing}
\label{sec:org650a32b}
Some network configurations can be vulnerable to deadlock.
Deadlock arises when a cyclic resources dependency occurs and the messages become blocked forever.
Dimension-order routing can avoid deadlock by removing one of the conditions: circular dependency.
\begin{itemize}
\item XY Routing:
\begin{itemize}
\item Forbid any Y to X turn
\item It is deterministic
\end{itemize}
\item West-first Routing:
\begin{itemize}
\item Forbid the west turns only
\item Partially adaptive
\end{itemize}
\end{itemize}
\section{Inter-Domain Routing}
\label{sec:org735079c}
\subsection{Border Gateway Protocol (BGP)}
\label{sec:org4186e5c}
BGP is the de facto inter-domain routing protocol for the Internet.
It supports route coordination across domains, referred as \textbf{``autonomous systems'' (AS)}.
BGP provides routers a means to:
\begin{description}
\item[{e(xternal)BGP}] Obtain subnet reachability information from neighbouring ASs.
\item[{i(nternal)BGP}] Propogate reachability information to all AS-internal routers.
\item Determine \emph{good} routes to other networks based on reachability information and policy.
\end{description}
See lecture slides for example.
\end{document}
