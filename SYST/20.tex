% Created 2019-04-07 Sun 11:01
% Intended LaTeX compiler: pdflatex
\documentclass[11pt]{article}
\usepackage[utf8]{inputenc}
\usepackage[T1]{fontenc}
\usepackage{graphicx}
\usepackage{grffile}
\usepackage{longtable}
\usepackage{wrapfig}
\usepackage{rotating}
\usepackage[normalem]{ulem}
\usepackage{amsmath}
\usepackage{textcomp}
\usepackage{amssymb}
\usepackage{capt-of}
\usepackage{hyperref}
\author{Adam Hawley}
\date{\today}
\title{Lecture 20: Network Layer \& Internet Protocol Continued}
\hypersetup{
 pdfauthor={Adam Hawley},
 pdftitle={Lecture 20: Network Layer \& Internet Protocol Continued},
 pdfkeywords={},
 pdfsubject={},
 pdfcreator={Emacs 26.1 (Org mode 9.2)}, 
 pdflang={English}}
\begin{document}

\maketitle
\tableofcontents


\section{Routing}
\label{sec:org8b89bfd}
Routing is needed for forwarding packets in a datagram (connectionless) network, or for establishing virtual circuits in a VC (connection oriented) network.
Routing algorithms or protocols create routing tables from which one may derive the neccesary forwarding tables.
These in turn, define the output port through which a packet will be forwarded.

Most routing protocols work only for 10s or 100s of nodes and hence they are referred to as \textbf{interior gateway protocols (IGPs)} or (\textbf{intra-domain routing protocols}).
To make them scale, internetworks employ a hierarchical routing structure based on \textbf{domains}.
\begin{itemize}
\item A \textbf{domain} is an internetwork where all routers are under a single administritative entity (e.g. university campus).
\item Each domain uses IGPs to route packages within its boundaries and uses gateway routes to forward packets to other domains (inter-domain routing).
\end{itemize}

\subsection{Graph Representation of Routing}
\label{sec:org5234b52}
Routing is a graph-theoretic problem and requires one to calculate the lowest-cost path between two nodes.
\begin{itemize}
\item Nodes are hosts, switches, routers or networks
\item Edges are network links, each associated with a cost.
\item Cost of a path is the sum of the costs of all traversed edges.
\end{itemize}
There are two main types of algorithms for solving the problem:
\begin{itemize}
\item Global routing: all routers have complete topology and link cost info --- "link state" algorithms.
\item Decentralised routing: router knows link costs to neighbours --- "distance vector" algorithms.
\end{itemize}
\end{document}
