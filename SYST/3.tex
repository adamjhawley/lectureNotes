\documentclass{article}

\title{Lecture 3: Processes}
\author{Adam Hawley}

\begin{document}

\maketitle

\section{Processes and Programs}

\begin{itemize}
	\item A \textbf{program} is an ordered set of specific operations for a computer to perform.
	\item A \textbf{process} is an abstraction of a running program.
\end{itemize}
A program is a passive entity stored on the disk including an executable file while a process is active and loaded into memory.

\section{Process Management}
Process management is one of the key functions of the OS, it includes:
\begin{itemize}
	\item Creating and deleting both user and system processes
	\item Suspending and resuming processes
	\item Deciding which process should run
	\item Providing mechanisms for process synchronisation
	\item Providing mechanisms for process communication
	\item Provising mechanisms for deadlock handling
\end{itemize}

\section{Process Structure}
A process is more than the program code, which is sometimes known as the text section.
It also includes the current activity such as the calue of the program counter and contents of the processor's registers.
It also includes the process stack, which contains temporary data (such as function parameters, return addresses and local variables) and a data section which contains global variables.
It may also include a heap, which is memory that is dynamically allocated during process run time.

See lecture slides for the process layout in memory.
In particular, slide 9 on the \textbf{Process Control Block}

\section{Process States}
One thing stored in the PCB is the process state.
Most OSs contain states which are similar or equivilent to the following:
\begin{itemize}
	\item \textbf{Ready (or Runnable)} --- scheduled to run but not currently using the processor
	\item \textbf{Running} --- currently using the processor
	\item \textbf{Waiting (or Blocked)} --- suspended waiting for some event to occur (such as a periodic clock tick or printer device to complete printing)
	\item \textbf{Terminated} --- the process has finished execution (and will be removed from the system ASAP
\end{itemize}

See slide 11 for a useful flow diagram of states.

\section{Process Lifecycle}
\begin{itemize}
	\item Processes are created
		\begin{itemize}
			\item at system initialisation
			\item by another process
			\item by a user (e.g double clicking an icon)
			\item batch job
		\end{itemize}
	\item On creation, OS will:
		\begin{itemize}
			\item allocate memory
			\item create PCB
			\item load program from disk to memory
			\item copy arguments to program's stack and itialise registers
		\end{itemize}
\end{itemize}

There are resource sharing options such as:
\begin{itemize}
	\item Parent and children process share all resources
	\item Children share subset of parent's resources
	\item Parent and child share no resources
\end{itemize}

And execution options:
\begin{itemize}
	\item Parent and children execute concurrently
	\item Parent waits until children terminate
\end{itemize}

\subsection{Ready and Running}
When a process is created it is places into a ready queue.
The scheduler decides which process to put into running state.
The \textbf{policy} makes the decision itself and the \textbf{mechanism} uses the information in the PCB to return to the selected process.

\subsection{Waiting/Blocked}
Processes issuing I/O requests are placed in a waiting state.
When the I/O completes, the process returns to the ready state.
If all processes are waiting, then the processor is idle until one of the processes becomes unblocked and ready.
This should be avoided at all costs.

\subsection{Terminated}
There are several different reasons a process can be moved into a terminated state:
\begin{itemize}
	\item Normal exit (voluntary)
	\item Error exit (voluntary)
	\item Fatal error (involuntary)
	\item Killed by another process (involuntary)
\end{itemize}

\subsection{Context Switch}
When CPU switches to another process, the system must save the state of the old process and load the saved state for the new process via a context switch.
Context of a process is represented in the PCB.

Context-switch time is pure overhead.
The system does no useful work while switching.
The more complex the OS and the PCB, the longer the context switch.

Switching is also dependent on hardware support.
Some hardware proved multiple sets of registers per CPU which means that there can be multiple contexts loaded at once.

See slide 18 for a context switch diagram
\end{document}
