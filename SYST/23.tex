% Created 2019-04-10 Wed 10:35
% Intended LaTeX compiler: pdflatex
\documentclass[11pt]{article}
\usepackage[utf8]{inputenc}
\usepackage[T1]{fontenc}
\usepackage{graphicx}
\usepackage{grffile}
\usepackage{longtable}
\usepackage{wrapfig}
\usepackage{rotating}
\usepackage[normalem]{ulem}
\usepackage{amsmath}
\usepackage{textcomp}
\usepackage{amssymb}
\usepackage{capt-of}
\usepackage{hyperref}
\author{Adam Hawley}
\date{\today}
\title{Lecture 23: System \& Network Security}
\hypersetup{
 pdfauthor={Adam Hawley},
 pdftitle={Lecture 23: System \& Network Security},
 pdfkeywords={},
 pdfsubject={},
 pdfcreator={Emacs 26.1 (Org mode 9.2)}, 
 pdflang={English}}
\begin{document}

\maketitle
\tableofcontents


\section{System Security \& Protection}
\label{sec:orgd9af8d1}
A computer consists of a collection of objects, hardware or software.
Each object has a unique name and can be accessed through a well-defined set of operations.
A \textbf{security policy} defines what it means to be secure for a particular system.
The \textbf{protection problem} is to ensure that each object is accessed correctly and only by those processes that are allowed to do so.

\section{Principles of Protection}
\label{sec:org93acc78}
\begin{itemize}
\item A \textbf{privilege} is the right to execute a particular operation on a given object.
\end{itemize}
One guiding principle is named the \textbf{principle of least privilege}, where:
\begin{itemize}
\item Programs, users and systems should be given just enough \textbf{privileges} to perform their tasks (this limits damage if the entity has a bug or gets abused).
\end{itemize}
Privileges can be one of:
\begin{itemize}
\item \textbf{Static} (During life of system, during life of process)
\item \textbf{Dynamic} (Changed by process as needed): \textbf{Domain switching privilege escalation}
\end{itemize}
\emph{``Need to know''} is a similar concept regarding access to data.

It is important to consider the \textbf{grain} aspect.
\begin{itemize}
\item \textbf{Rough-Grained}: Management is easier,simpler but least privilege is now done in large chunks.
\begin{itemize}
\item For example, traditional Unix processes either have abilities of the associated user or of the root.
\end{itemize}
\item \textbf{Fine-Grained}: More complex, more overhead but more protective.
\begin{itemize}
\item For examlpe, ACL (Access Control List) or RBAC (Role Based Access Control).
\end{itemize}
\end{itemize}
Privilege management is commonly supported  by the notion of domains which can be user, process, procedure etc.

\section{Domain Structure}
\label{sec:orgd318d31}
\begin{itemize}
\item Access-Right = \texttt{<object-name, rights-set>}
\begin{itemize}
\item \texttt{rights-set} is a subset of all valid operations that can be performed on the object.
\end{itemize}
\item Domain = set of access-rights.
\end{itemize}
A process, at any point in time, is associated with one domain but can switch domains in a controlled way.
Domains can overlap.

See slide 7 for Unix example.

\section{Security}
\label{sec:org18a8e74}
A system is said to be secure if resources are used and accessed as intended under all circumstances.
\begin{description}
\item[{Intruders}] Attempt to breach security
\item[{Threat}] Potential security violations
\item[{Attack}] Attempt to breach security (can be accidental or malicious but easier to protect against accidental).
\end{description}

\subsection{Security Violation Categories}
\label{sec:org5c5aee5}
\begin{description}
\item[{Breach of Confidentiality}] Unauthorised reading of data.
\item[{Breach of Integrity}] Unauthorised modification of data.
\item[{Breach of Availability}] Unauthorised destruction of data.
\item[{Theft of Service}] Unauthorised use of resources.
\item[{Denial of Service (DoS)}] Prevention of legitimate use.
\end{description}

\subsection{Security Violation Methods}
\label{sec:orgfcdb3ac}
\begin{description}
\item[{Masquerading (breach authentication)}] Pretending to be an authorised user to escalate privileges.
\item[{Replay Attack}] As is or with \textbf{message modification}.
\item[{Man-in-the-middle Attack}] Intruder sits in data flow, masquerading as sender to receiver and vice versa.
\item[{Interception}] Intercept an already-established session to bypass authentication (e.g. sniffing, hijacking, covert channel).
\end{description}

\subsection{Security Measure Levels}
\label{sec:org4542934}
It is impossible too have absolute security, but make cost to perpetrator sufficiently high to deter most intruders.
Security must occur at four levels to be effective:
\begin{description}
\item[{Physical}] Control access to data ceters, servers, connected terminals
\item[{Human}] Avoid social engineering, phishing, dumpster diving
\item[{Operating System}] Protection mechanisms
\item[{Network}] Encryption, firewalls, blacklisting
\end{description}
Security is as weak as the weakest link in the chain.

\section{Examples}
\label{sec:orgd3bb893}
See lecture for examples on \textbf{Meltdown} and \textbf{DNS Spoofing}.
\end{document}
