% Created 2019-03-31 Sun 21:32
% Intended LaTeX compiler: pdflatex
\documentclass[11pt]{article}
\usepackage[utf8]{inputenc}
\usepackage[T1]{fontenc}
\usepackage{graphicx}
\usepackage{grffile}
\usepackage{longtable}
\usepackage{wrapfig}
\usepackage{rotating}
\usepackage[normalem]{ulem}
\usepackage{amsmath}
\usepackage{textcomp}
\usepackage{amssymb}
\usepackage{capt-of}
\usepackage{hyperref}
\author{Adam Hawley}
\date{\today}
\title{Lecture 18: Physical \& Data Link Layers}
\hypersetup{
 pdfauthor={Adam Hawley},
 pdftitle={Lecture 18: Physical \& Data Link Layers},
 pdfkeywords={},
 pdfsubject={},
 pdfcreator={Emacs 26.1 (Org mode 9.2)}, 
 pdflang={English}}
\begin{document}

\maketitle
\tableofcontents


\section{Physical Layer}
\label{sec:org3a6b9a1}
Deals with the transmission of bits over physical medium (twisted pair/coaxial cables, fibre optics, wireless etc.).
The layers usually have a standardised interface to transmission media (e.g. how 1/0 signals are carried over a link).

Requirements of the physical layer include both \textbf{synchronisation} and \textbf{flow control} as well as multiplexing (FDM (frequency division multiplexing), TDM (time division multiplexing) and CDM (code division multiplexing)).
Multiplexing is where you use the same connection for multiple transmissions.

\section{Data Link Layer}
\label{sec:orgc49f324}
Deals with the framing of raw data, as well as flow control and error correction (i.e. if they cannot be solved at the physical level).
The data linke layer is often divided into two sublayers:
\begin{itemize}
\item \textbf{Logical Link Control (LLC)}: Multiplex protocols running over the data link layer, error and flow control.
\item \textbf{Media Access Control (MAC)}: Control channel access, append and check FCS (frame chack sequence), discard malformed frames, addressing.
\end{itemize}

\subsection{Example: Ethernet}
\label{sec:org846e67b}
\begin{itemize}
\item Standard: IEEE 802.3
\end{itemize}

Uses CSMA/CD local area network technology:
\begin{itemize}
\item Multiple-Access (MA): Several nodes are connected to the same cable (cf. data bus).
\item Carrier Sense (CS): A node can distinguish between a busy and an idle link.
\item Collision Detect (CD): A node listens when it transmits a frame in order to detect whether the frame interferes (collides) with a frame transmitted by another node.
\end{itemize}
Multiple Ethernet segments are joined by \textbf{repeaters} that forward the signals.
In an Ethernet, every signal is propagated in all directions over the entire network, even crossing repeater boundaries.
\begin{itemize}
\item Problem: All hosts compete for access to the same link; they are said to be in the same \textbf{collision detection}.
\end{itemize}
The problem is solved by intelligently partitioning the \textbf{collision domain} using the following:
\begin{description}
\item[{Hub}] Multiway repeater supporting several point-to-point segments - still only one collision domain.
\item[{Bridge}] Each port is connected to a different collision domain. Transmissions within separate domains are allowed to happen in parallel.
\item[{Switch}] Frames are sent only to their destinations --- no collisions.
\end{description}
\end{document}
