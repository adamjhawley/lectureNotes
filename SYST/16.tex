% Created 2019-03-27 Wed 16:41
% Intended LaTeX compiler: pdflatex
\documentclass[11pt]{article}
\usepackage[utf8]{inputenc}
\usepackage[T1]{fontenc}
\usepackage{graphicx}
\usepackage{grffile}
\usepackage{longtable}
\usepackage{wrapfig}
\usepackage{rotating}
\usepackage[normalem]{ulem}
\usepackage{amsmath}
\usepackage{textcomp}
\usepackage{amssymb}
\usepackage{capt-of}
\usepackage{hyperref}
\author{Adam Hawley}
\date{\today}
\title{Lecture 16: Introduction to Networking}
\hypersetup{
 pdfauthor={Adam Hawley},
 pdftitle={Lecture 16: Introduction to Networking},
 pdfkeywords={},
 pdfsubject={},
 pdfcreator={Emacs 26.1 (Org mode 9.2)}, 
 pdflang={English}}
\begin{document}

\maketitle
\tableofcontents


\section{Computer Networks}
\label{sec:orgdef3cb1}
\begin{itemize}
\item An \uline{interconnection} of \uline{autonomous} computers
\end{itemize}
They should be ableto exchange information (connected via some medium such as copper wire, microwaves etc.).
They also lack a centralised control.

Motivation for developing networks:
\begin{itemize}
\item Resource sharing
\item Remote access and communication
\item Reliability and redundancy
\item Economic advantages
\end{itemize}

\section{Examples of Networks}
\label{sec:org0270caa}
\begin{itemize}
\item Personal Area Network (PAN)
\item Local Area Network (LAN)
\item Wide Area Network (WAN)
\item Internet
\end{itemize}

\section{Network Topologies}
\label{sec:org443b66f}
Examples:
\begin{itemize}
\item Bus
\item Ring
\item Star
\item Mesh
\end{itemize}

\section{Connectivity}
\label{sec:orgbb7b840}
At the lowest level \textbf{nodes} are connected by some physical medium called a \textbf{link}.
In the course, if it is not otherwise stated then it should be assumed that links are in a \textbf{duplex mode} (rather than simplex) where data can be transported in both directions.
Physical (direct) connections can be \textbf{point-to-point} or \textbf{multiple-access}.

\section{Indirect Connectivity}
\label{sec:org62cd7e8}
\begin{itemize}
\item Physical connections are often not direct.
\end{itemize}
When using indirect connectivity there is usually a \textbf{switch node} involved which is a node attached to multiple point-to-point links.
There are two main classes of indirect connectivity: circuit-switched networks and packet-switched networks.

\subsection{Circuit-Switched Networks}
\label{sec:org3f6c490}
\begin{itemize}
\item A circuit must be established before data is sent.
\item All blocks of a large message are always sent via the same route e.g. telephone network.
\item Advantage: Time for data to be delivered is predictable once the circuit is set.
\end{itemize}
SEE SLIDES FOR DEMO

\subsection{Packet-Switched Networks}
\label{sec:orgd794909}
\begin{itemize}
\item Handle blocks of data (\emph{packets})
\item Different ways to handle packets along the way e.g. store-and-forward, wormhole etc.
\end{itemize}
SEE SLIDES FOR DEMO

\section{Interconnection of Networks}
\label{sec:org5952d75}
\emph{An internet} is an interconnection of independent networks.
The currently operational TCP/IP internet is called \emph{The Internet}.

\section{Addressing, Routing \& Casting}
\label{sec:orgdbe2976}
\textbf{Addresses} of nodes are needed to uniquely identify recipient nodes of messages.
Switches use addresses to describe how to forward (to \emph{route}) a message towards its destination.
Depending on whether a source node (sender) sense a message to a single destination node (receiver), or to multiple or even all nodes, one speaks of \textbf{unicasting, multicasting} or \textbf{broadcasting}, respectively.

\section{Reliability}
\label{sec:orgbeeb098}
\subsection{Errors At The Bit Level}
\label{sec:orgbd5cb0c}
Corruption of single bits is very unlikely in copper cable (1:1,000,000) and optical fibre (1:10,000,000,000,000).
The corruption often occurs in clusters (burst errors) and are caused by power surges, lightning or microwaves.
This kind of corruption can be detected and sometimes corrected.

\subsection{Errors At The Packet Level}
\label{sec:orgffac70e}
This is when complete packets are lost and can be because of non-correctable bit errors.
These errors also appear at congested switches because it can be difficult to distinguish between a packet that is indeed lost and one that is merely late in arriving at its destination.

\subsection{Errors At The Node/Link Level:}
\label{sec:org9803586}
\begin{itemize}
\item Due to a cut physical link
\item Caused by hardware or software crashes
\item Are eventually (after a \emph{long} time) corrected
\item It is difficult to distinguish between a cut and an error-prone link, or between a failed and slow node.
\end{itemize}
\textbf{Note}: Packet-switched networks have the abililty to possibly route around a failed node or link.

\section{Common Units in Networking}
\label{sec:orgdb20f1d}
\begin{itemize}
\item Units in computing are commonly shown as powers of 2 e.g. GB = 2\textsuperscript{30} bytes = 2\textsuperscript{33} bits.
\item Units of engineering are commonly shwon as powers of 10. e.g. Gbps: giga bit per second = 10\textsuperscript{9} or 2\textsuperscript{30} bits per second
\end{itemize}

\section{Performance}
\label{sec:org8dc7222}
The efficiency of applications running over the network often depends on the efficiency of the network itself.
There are two key performance metrics:
\begin{itemize}
\item Bandwidth (\emph{transmission speeds}) How many bytes per second can I move
\item Latency (\emph{transmission delays}) What is the time taken from beginning to end of a transfer
\end{itemize}
These can both be measured at the levels of individual links as well as end-to-end channels.

\subsection{Bandwidth}
\label{sec:org2ab1e60}
Number of bits that can be transmitted over the link/channel in a certain period of time.
Bandwidth is usually measured in bps (bits per second).
Factors which influence bandwidth:
\begin{itemize}
\item The physical/electrical characteristics of links
\item The software overhead needed for handling/transforming each bit of data (in the case of channels)
\item The load of networks (in the case of channels)
\end{itemize}

\subsection{Latency (Delay)}
\label{sec:org426f532}
Latency in networks consists of three components:
\begin{enumerate}
\item Speed-of-light \textbf{propogation} delay (the speed of light is 2.3\texttimes{} 10\textsuperscript{8} m/s in cable and 2.0 \texttimes{} 10\textsuperscript{8} m/s in fibre.
\item Time needed to \textbf{transmit} a unit of data (usually a packet).
\item \textbf{Queuing} delays inside the network (at switches).
\end{enumerate}
Formula:
\begin{equation}
Latency = Propogation + Transmission + Queuing
\end{equation}
\begin{equation}
Propogation = \frac{Distance}{SpeedOfLightInMedium}
\end{equation}
\begin{equation}
Transmission = \frac{Size}{Bandwidth}
\end{equation}

\subsection{Latency Vs. Bandwidth}
\label{sec:org176c629}
To which degree latency and bandwidth dominate performance depends on the application of the program.
For example, a digital library program that is asked to fetch a 25MB image, with a 10Mbps channel needing about 20s for transmission means that other latency which is usually in the range of ms, does not matter.
While a transatlantic network with a bandwidth of 45Mbps will need around 0.366ms for transmission of a 2KB message but the other latency which is in the range of 100ms dominates.

\subsection{Delay \texttimes{} Bandwidth Product}
\label{sec:org0c5e0c8}
How many bits can be sent before the first one arrives at the receiver?
\begin{itemize}
\item Answer: \#bits = dela \texttimes{} bandwidth
\end{itemize}
The product also indicated how many bits can be \emph{stored} in a link.

\subsection{Application Performance Needs}
\label{sec:org0cd8106}
Often applications want as much bandwidth as possible but sometimes other factors are more important.
For example, the human eye can only process about 30 fps so bandwidth more than this is not useful.
What matters is controlling jitter, i.e. the variation in latency (buffers storing incoming packets should not underflow or overflow).
Lower and upper bounds on delays are needed.
\end{document}
