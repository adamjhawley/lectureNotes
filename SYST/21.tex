% Created 2019-04-08 Mon 15:48
% Intended LaTeX compiler: pdflatex
\documentclass[11pt]{article}
\usepackage[utf8]{inputenc}
\usepackage[T1]{fontenc}
\usepackage{graphicx}
\usepackage{grffile}
\usepackage{longtable}
\usepackage{wrapfig}
\usepackage{rotating}
\usepackage[normalem]{ulem}
\usepackage{amsmath}
\usepackage{textcomp}
\usepackage{amssymb}
\usepackage{capt-of}
\usepackage{hyperref}
\author{Adam Hawley}
\date{\today}
\title{Lecture 21: Transport Layer}
\hypersetup{
 pdfauthor={Adam Hawley},
 pdftitle={Lecture 21: Transport Layer},
 pdfkeywords={},
 pdfsubject={},
 pdfcreator={Emacs 26.1 (Org mode 9.2)}, 
 pdflang={English}}
\begin{document}

\maketitle
\tableofcontents


\section{Introduction}
\label{sec:orgb58c24f}
The transport layer provides \textbf{end-to-end connectivity} in terms of a \textbf{transport protocol} (end-to-end protocol).
The underlying network layer usually only provides a \textbf{best-effort host-to-host service} (e.g. IP):
\begin{itemize}
\item messages are dropped (due to congestion)
\item messages are re-ordered
\item messages are delivered several times (problem of duplicates)
\item messages are limited to some finite size
\item messages are delivered after some long delay
\end{itemize}
Different transport protocols address (some of) these limitations by offering different services:
\begin{itemize}
\item Simple (application) demultiplexing service (\textbf{User Datagram Protocol})
\item Reliable Byte-Stream Service (\textbf{Transmission Control Protocol})
\end{itemize}
\end{document}
