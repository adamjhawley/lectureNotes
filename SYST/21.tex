% Created 2019-04-09 Tue 11:46
% Intended LaTeX compiler: pdflatex
\documentclass[11pt]{article}
\usepackage[utf8]{inputenc}
\usepackage[T1]{fontenc}
\usepackage{graphicx}
\usepackage{grffile}
\usepackage{longtable}
\usepackage{wrapfig}
\usepackage{rotating}
\usepackage[normalem]{ulem}
\usepackage{amsmath}
\usepackage{textcomp}
\usepackage{amssymb}
\usepackage{capt-of}
\usepackage{hyperref}
\author{Adam Hawley}
\date{\today}
\title{Lecture 21: Transport Layer}
\hypersetup{
 pdfauthor={Adam Hawley},
 pdftitle={Lecture 21: Transport Layer},
 pdfkeywords={},
 pdfsubject={},
 pdfcreator={Emacs 26.1 (Org mode 9.2)}, 
 pdflang={English}}
\begin{document}

\maketitle
\tableofcontents


\section{Introduction}
\label{sec:org41c3064}
The transport layer provides \textbf{end-to-end connectivity} in terms of a \textbf{transport protocol} (end-to-end protocol).
The underlying network layer usually only provides a \textbf{best-effort host-to-host service} (e.g. IP):
\begin{itemize}
\item messages are dropped (due to congestion)
\item messages are re-ordered
\item messages are delivered several times (problem of duplicates)
\item messages are limited to some finite size
\item messages are delivered after some long delay
\end{itemize}
Different transport protocols address (some of) these limitations by offering different services:
\begin{itemize}
\item Simple (application) demultiplexing service (\textbf{User Datagram Protocol})
\item Reliable Byte-Stream Service (\textbf{Transmission Control Protocol})
\end{itemize}

\section{User Datagram Protocol (UDP)}
\label{sec:org7f3376c}
UDP is a simple extension of the host-to-host delivery service to a process-to-process communication service.
It lets multiple application processes share the same network.
Using an \textbf{abstract locator}, a process and a \textbf{port} are identified.
The network delivers messages to the port and a destination receives messages from the port.
UDP uses a 16-bit numbering scheme to identify ports.
Processes are identified by a \textbf{demultiplexing key} of the form \texttt{<port,host>}.
However, UDP does not support any other mechanisms such as flow control or reliable/ordered delivery.

\subsection{Implementation of UDP}
\label{sec:orgc00e1bf}
\subsubsection{How to learn of ports?}
\label{sec:org686f5f0}
\begin{itemize}
\item Clients send initial messages to specific servers via some \emph{well-known} (published) ports, e.g. port 80 for HTTP.
\item A well-known port is used to agree on some other port that will be used subsequently.
\end{itemize}
\subsubsection{How are ports implemented?}
\label{sec:org3420aa9}
\begin{itemize}
\item Via message queues.
\end{itemize}

\section{Transmission Control Protocol (TCP)}
\label{sec:org69efa5b}
TCP implements a reliable, connection-oriented byte-stream service with guaranteed in-order delivery, flow control and network congestion control.
It supports full-duplex connectivity, i.e. pairs of byte streams between two processes, one stream in each direction.
It also offers a \textbf{demultiplexing} mechanism, similar to UDP.

\subsection{The Segment Format of TCP}
\label{sec:org1e8fa1b}
While TCP allows senders to write bytes \emph{into a connection} and receivers to read bytes \emph{out of a connection}, it transmits byte sequences in reasonably sized packets, called \textbf{segments}.

\subsection{When to Send A Segment?}
\label{sec:org23e9dd1}
\begin{itemize}
\item When the \textbf{maximum segment size is reached}.
\begin{itemize}
\item Typically set to about the MTU (maximum transmission unit) of the directly connected network, such that the local IP does not need to fragment segments.
\end{itemize}
\item When the process explicity invokes a \textbf{push operation} to flush the TCP buffer of unsent bytes.
\item When some \textbf{periodic timer} fires (and flushes the TCP buffer).
\end{itemize}

\subsection{How Are TCP Connections Uniquely Identified?}
\label{sec:org79e9abc}
Via a demultiplexingkey that is of the form:
\begin{itemize}
\item \textasciitilde{}<SrcPort, SrcIPAddr, DstPort, DstIPAddr>
\end{itemize}
But different incarnations of a connection are possible since connections are established and torn down.

\section{Connection Establishment \& Termination}
\label{sec:orgb4e581a}
\begin{itemize}
\item Connection Establishment (\textbf{asymmetric Activity}):
\begin{enumerate}
\item Server performs a \textbf{passive open} (create buffer to read from).
\item Client does an \textbf{active open} to the server.
\item Two sides echange messages to establish the connection.
\end{enumerate}
\item Connection Termination (\textbf{Symmetric Activity}):
\begin{enumerate}
\item Participant closes one direction of the connection.
\item Connection termination messages are exchanged.
\item Other participant might leave the other direction open.
\end{enumerate}
\end{itemize}
See slide 13 for \textbf{TCP State Diagram}.

\subsection{Three-Way Handshake Algorithm for Connection Establishment}
\label{sec:org7000e40}
The goal of the three-way handshake is to agree on some parameters, such as starting sequence number (for each direction).

Notes:
\begin{itemize}
\item Due to different incarnations, it is a good idea to start with \emph{random} sequence numbers instad of 0. Otherwise an acknowledgement from a previous incarnation may be taken as an acknowledgement for the current incarnation.
\item The acknowledgement field actually identifies the \textbf{next expected sequence number}.
\end{itemize}

\section{Flow Control \& Congestion Control}
\label{sec:orgff0065f}
Two different factors can limit the rate at which a source sends data:
\begin{itemize}
\item The inability of the destination to accept new data.
\begin{itemize}
\item Techniques that address this are referred to as \textbf{flow control}.
\end{itemize}
\item The number of packets within the subnet.
\begin{itemize}
\item Techniques that address this are referred to as \textbf{congestion control}.
\end{itemize}
\end{itemize}
TCP standard includes both techniques.
Some protocol stacks also apply such technique at lower layers (e.g. flow control at the data link level, congestion control at the network level.

\subsection{Flow Control}
\label{sec:org05f055d}
The key functionality of a flow control algorithm is that the sender and receiver have finite buffer space and process data at different rates.
Rather than waiting for acknowledgement on every message (which would impact performance greatly) we can send several segments at once.
However, the segments should not be removed from the buffer until the acknowledgement is received.
This is called a sliding window protocol because there is an upper-bound on \emph{un-ACKed} segments, called a window which limits the amount of buffer space required at sender and receiver.
A bigger window means more flexibility but also a greater risk that failures remain undected.

The sender knows:
\begin{itemize}
\item Which data has been sent and acknowledged
\item The receivers buffer size
\item Its own buffer size
\end{itemize}
The sender creates a window stretching from:
\begin{itemize}
\item Last data acknowledged (X), to
\item X + send buffer size
\end{itemize}
If the sender sends data beyond this it can't re-send in case of failure.
By a similar process the receiver calculates its available window (\textbf{AdvertisedWindow}) and informs the sender of it.
The sender does not send information that the receiver cannot handle, i.e. buffer.

\subsection{Congestion Control}
\label{sec:org8dade4a}
Typical congestion control techniques at the transport layer.
\begin{itemize}
\item Rate Control: Source injection rate is explicitly controlled based on feedback from either the network and/or the receiver (e.g. keep increasing injection rate until packets are dropped / ACKs are not received).
\item Admission Control/ Traffic Policing: Only allow connections in if the network can handle them and make sure that admitted sessions do not send them at too high of a rate.
\end{itemize}

\section{Retransmission}
\label{sec:orgeb37643}
Retransmission (from sender) of unacknowledged segments after some timeout has expired.
This requires a timer but there are two main ways of choosing the value of said timer:
\begin{itemize}
\item Fixed value needs knowledge of network behaviour
\begin{itemize}
\item Does not respond to changing network conditions
\item Too small, network flooded with retransmissions
\item Too large and the receiver stalls
\end{itemize}
\item Adaptive timer
\begin{itemize}
\item Difficult to monitor \textbf{Re-Transmission Timer (RTT)} with cumulative acknowledgements.
\item Network conditions may change faster than we can adapt.
\end{itemize}
\end{itemize}

\subsection{Weighted (Moving) Average}
\label{sec:org8828f22}
\begin{equation}
\label{eq:estrtt}
EstRTT = \alpha \times EstRTT + \beta \times SampleRTT
\end{equation}
Where:
\begin{itemize}
\item \(\alpha + \beta = 1\)
\item \(0.8 \le \alpha \le 0.9\)
\item \(0.1 \le \beta \le 0.2\)
\end{itemize}
The weighted average is calculated with the following steps:
\begin{enumerate}
\item Measure \texttt{SampleRTT} for each segment/ACK pair.
\item Compute weighted average of RTT (see equation \eqref{eq:estrtt})
\item Set timeout based on \texttt{EstRTT}
\begin{itemize}
\item \(TimeOut = 2 \times EstRTT\)
\end{itemize}
\end{enumerate}
\end{document}
