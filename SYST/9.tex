% Created 2019-02-06 Wed 15:15
% Intended LaTeX compiler: pdflatex
\documentclass[11pt]{article}
\usepackage[utf8]{inputenc}
\usepackage[T1]{fontenc}
\usepackage{graphicx}
\usepackage{grffile}
\usepackage{longtable}
\usepackage{wrapfig}
\usepackage{rotating}
\usepackage[normalem]{ulem}
\usepackage{amsmath}
\usepackage{textcomp}
\usepackage{amssymb}
\usepackage{capt-of}
\usepackage{hyperref}
\author{Adam Hawley}
\date{\today}
\title{Lecture 9: Deadlock Pt.2}
\hypersetup{
 pdfauthor={Adam Hawley},
 pdftitle={Lecture 9: Deadlock Pt.2},
 pdfkeywords={},
 pdfsubject={},
 pdfcreator={Emacs 26.1 (Org mode 9.2)}, 
 pdflang={English}}
\begin{document}

\maketitle
\tableofcontents


\section{Resource-Request Algorithm for Process P\textsubscript{i}}
\label{sec:org81dc827}
Let \texttt{Request\_i[...]} be the request vector for process \textbf{P\textsubscript{i}}.
\begin{description}
\item[{\texttt{Request\_i[j] = k}}] Process \textbf{P\textsubscript{i}} wants \textbf{k} instances of resource type \textbf{R\textsubscript{j}}.
\end{description}


\begin{enumerate}
\item If \texttt{Request\_i} \(\le\) \texttt{Need\_i} go to step 2, otherwise raise error condition since process has exceeded its maximum claim.
\item If \texttt{Request\_i} \(\le\) \texttt{Available}, go to step 3, otherwise \textbf{P\textsubscript{i}} must wait since resources are not available.
\item Pretend to allocate requested resources to \textbf{P\textsubscript{i}} by modifying the state as follows:
\end{enumerate}

\begin{verbatim}
Available = Available - Request;
Allocation_i = Allocation_i + Request_i;
Need_i = Need_i - Request_i;
\end{verbatim}

\begin{itemize}
\item \texttt{If safe} \(\Rightarrow\) the resources are allocated to \textbf{P\textsubscript{i}}
\item \texttt{If unsafe} \(\Rightarrow\) \textbf{P\textsubscript{i}} must wait, and the old resource-allocation state is restored.
\end{itemize}

\section{Deadlock Detection}
\label{sec:org0a58a4c}
\begin{itemize}
\item Allow system to enter deadlock state
\item Detection algorithm
\begin{itemize}
\item Single Instance of a Resource Type
\item Multiple Instances of a Resource Type
\end{itemize}
\item Recovery Scheme
\end{itemize}
\subsection{Detection Algorithm for Single Instances of a Resource Type}
\label{sec:org7c77c54}
Maintain a \textbf{wait-for} graph.
In a \textbf{wait-for} graph, nodes are only processes, there are no resources.

\begin{itemize}
\item \textbf{P\textsubscript{i}} \(\rightarrow\) \textbf{P\textsubscript{j}} if \textbf{P\textsubscript{i}} is waiting for \textbf{P\textsubscript{j}}
\end{itemize}

To convert between a resource-allocation graph and a wait-for graph, just join processes where one is holding a resource that the other needs.
Cycles in a wait-for diagram show deadlock.

Periodically invoke an algorithm that searches for a cycle in the graph.
An algorithm to detect a cycle in a graph requires an order of \textbf{n\textsuperscript{2}} operations, where \textbf{n} is the number of vertices in a graph, this is why this algorithm is only invoked periodically.

\subsection{Detection Algorithm for Multiple Instances of a Resource Type}
\label{sec:orgf38b62c}
Where \emph{n} = number of processes and \emph{m} = number of resource types:

\begin{description}
\item[{Available}] Vector of length \emph{m}. If \texttt{Available[j] == k}, then there are \emph{k} instances of resource type \textbf{R\textsubscript{j}} available.
\item[{Allocation}] \emph{n} x \emph{m} matrix. If \texttt{Allocation[i,j] == k}, then \textbf{P\textsubscript{i}} is currently allocated \emph{k} instances of \textbf{R\textsubscript{j}}.
\item[{Request}] \emph{n} x \emph{m} matrix that indicates the current request of each process. If \texttt{Request[i,j] == k}, then process \textbf{P\textsubscript{i}} is requesting \emph{k} additional instances of resource type \textbf{R\textsubscript{j}}.
\end{description}

See lecture 8 for details of algorithm. 

\subsection{Outlining the Detection Algorithm Stage}
\label{sec:org2b2a75d}
If a deadlock is  detected, we must abort (rollback) some of the processes involved in the deadlock.
Need to decide when and how often to invoke the deadlock detection algorthm which depends on the following:
\begin{itemize}
\item How often a deadlock is likely to occur?
\item How many processes will need to be rolled back?
\begin{itemize}
\item (One for each disjoint cycle)
\end{itemize}
\end{itemize}

If a detection algorithm is invoked arbitrarily, there may be many cycles in the resource graph and so we would not be able to tell which of the many deadlocked processes \emph{caused} the deadlock.

\section{Deadlock Recovery}
\label{sec:org39ec003}
\subsection{Process Termination}
\label{sec:orge38a333}
\begin{itemize}
\item Abort all deadlocked processes
\item Abort one process at at time until the deadlock cycle is eliminated.
\end{itemize}

In which order should we choose to abort?
\begin{itemize}
\item Priority of the process
\item How long process has computed, and how much longer to completion
\item Resources the process has used
\item Resources process needs to complete
\item How many processes will need to be terminated
\item Is the process interactive or batch?
\end{itemize}

\subsection{Resource Preemption}
\label{sec:orgee2757c}
\begin{description}
\item[{Selecting a victim}] Minimise cost
\item[{Rollback}] Return to some safe state, restart process for that state
\item[{Starvation}] If the same process is always picked as a victim then it could suffer from starvation. To avoid this, we can include the number of rollbacks in cost factor.
\end{description}



\section{Process Management - Conclusion}
\label{sec:org8fdf03f}
What we have covered:
\begin{itemize}
\item Processes, threads and their life cycle
\item Scheduling
\item Synchronising
\item Deadlock
\item Hardware Support
\end{itemize}
In the next section: memory management.
\end{document}
